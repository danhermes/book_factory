% Options for packages loaded elsewhere
\PassOptionsToPackage{unicode}{hyperref}
\PassOptionsToPackage{hyphens}{url}
\documentclass[
]{article}
\usepackage{xcolor}
\usepackage[top=0.75in,bottom=0.75in,left=0.65in,right=0.65in]{geometry}
\usepackage{amsmath,amssymb}
\setcounter{secnumdepth}{-\maxdimen} % remove section numbering
\usepackage{iftex}
\ifPDFTeX
  \usepackage[T1]{fontenc}
  \usepackage[utf8]{inputenc}
  \usepackage{textcomp} % provide euro and other symbols
\else % if luatex or xetex
  \usepackage{unicode-math} % this also loads fontspec
  \defaultfontfeatures{Scale=MatchLowercase}
  \defaultfontfeatures[\rmfamily]{Ligatures=TeX,Scale=1}
\fi
\usepackage{lmodern}
\ifPDFTeX\else
  % xetex/luatex font selection
\fi
% Use upquote if available, for straight quotes in verbatim environments
\IfFileExists{upquote.sty}{\usepackage{upquote}}{}
\IfFileExists{microtype.sty}{% use microtype if available
  \usepackage[]{microtype}
  \UseMicrotypeSet[protrusion]{basicmath} % disable protrusion for tt fonts
}{}
\makeatletter
\@ifundefined{KOMAClassName}{% if non-KOMA class
  \IfFileExists{parskip.sty}{%
    \usepackage{parskip}
  }{% else
    \setlength{\parindent}{0pt}
    \setlength{\parskip}{6pt plus 2pt minus 1pt}}
}{% if KOMA class
  \KOMAoptions{parskip=half}}
\makeatother
\usepackage{longtable,booktabs,array}
\usepackage{calc} % for calculating minipage widths
% Correct order of tables after \paragraph or \subparagraph
\usepackage{etoolbox}
\makeatletter
\patchcmd\longtable{\par}{\if@noskipsec\mbox{}\fi\par}{}{}
\makeatother
% Allow footnotes in longtable head/foot
\IfFileExists{footnotehyper.sty}{\usepackage{footnotehyper}}{\usepackage{footnote}}
\makesavenoteenv{longtable}
\usepackage{graphicx}
\makeatletter
\newsavebox\pandoc@box
\newcommand*\pandocbounded[1]{% scales image to fit in text height/width
  \sbox\pandoc@box{#1}%
  \Gscale@div\@tempa{\textheight}{\dimexpr\ht\pandoc@box+\dp\pandoc@box\relax}%
  \Gscale@div\@tempb{\linewidth}{\wd\pandoc@box}%
  \ifdim\@tempb\p@<\@tempa\p@\let\@tempa\@tempb\fi% select the smaller of both
  \ifdim\@tempa\p@<\p@\scalebox{\@tempa}{\usebox\pandoc@box}%
  \else\usebox{\pandoc@box}%
  \fi%
}
% Set default figure placement to htbp
\def\fps@figure{htbp}
\makeatother
\setlength{\emergencystretch}{3em} % prevent overfull lines
\providecommand{\tightlist}{%
  \setlength{\itemsep}{0pt}\setlength{\parskip}{0pt}}
\usepackage{bookmark}
\IfFileExists{xurl.sty}{\usepackage{xurl}}{} % add URL line breaks if available
\urlstyle{same}
\hypersetup{
  hidelinks,
  pdfcreator={LaTeX via pandoc}}

\title{ChatGPT for Business}
\author{Dan Hermes}
\date{\today}

\begin{document}
\maketitle

\pagenumbering{roman}
\tableofcontents
\clearpage
\pagenumbering{arabic}

\pandocbounded{\includegraphics[keepaspectratio]{D:/My Documents/Lexicon Systems/AI/Writing/Prompt Power/cover/Dan_Hermes_Prompt_Power - FINAL.png}}

gellitos

\section{Prompt Power:}\label{prompt-power}

\subsection{Learn to Create ChatGPT
Prompts}\label{learn-to-create-chatgpt-prompts}

\paragraph{by Dan Hermes}\label{by-dan-hermes}

Talk to My Agent - Boston, Massachusetts

gellitos

Copyright 2025 Dan Hermes

All rights reserved

No part of this publication may be reproduced, stored or introduced into
a retrieval system, or transmitted, in any form, or by any means
(electronic, mechanical, photocopying, recording, or otherwise), without
prior permission of the publisher. Requests for permission should be
directed to talktomyagent@lexicon.systems.

Author contact: dan@lexicon.systems

Hermes, Dan, 1970- Prompt Power: Learn to Create ChatGPT Prompts First
Edition ISBN 979-8-9988402-0-3 (paperback)

gellitos

To begin wisely is to ask well.

gellitos

gellitos

\section{Table of Contents}\label{table-of-contents}

About the Author

Acknowledgements

Introduction

Chapter 1: The Request

Chapter 2: Give Examples

Chapter 3: How to Format Your Output

Chapter 4: What's the Context?

Chapter 5: Your Assistant's Role

Chapter 6: What are the Constraints?

Chapter 7: How to Prompt Powerfully

Appendix: 500 Power Prompts

Appendix: Prompt Outside the Box

Index

Epilogue

Teaser

About Lexicon Systems

gellitos

PIC?

\section{About the Author}\label{about-the-author}

\textbf{Dan Hermes} works at the intersection of AI agents and empowered
teams, driving innovation in Generative AI (GenAI), Large Language
Models (LLMs), and software development. With a passion for making
cutting-edge ideas into impactful solutions, he specializes in designing
and delivering scalable, budget-conscious products that shape industries
and elevate user experiences. His expertise spans fine-tuning LLMs,
Retrieval-Augmented Generation (RAG), embedding pipelines, and agentic
architecture.

Mr.~Hermes has led and architected projects at the forefront of GenAI
and AI-driven applications, guiding enterprises and startups alike in
harnessing the transformative potential of AI. His career includes
authoring published books and articles on AI, mobile development, and
architecture. His thought leadership has appeared in \emph{IBM Mobile
Business Insights} and \emph{Microsoft's MSDN Magazine}, and he has
delivered talks at events such as Microsoft Ignite and IBM Think, along
with courses for edX and Pluralsight.

Over the past decade, Mr.~Hermes has led global software and data
teams---most recently at Avanade, an Accenture and Microsoft
company---where he architected advanced systems for airlines, banks, and
municipalities. For more than 20 years, he has run Lexicon Systems, a
consultancy serving clients including Microsoft, Fidelity Investments,
Thermo Fisher Scientific, DraftKings, and the FAA.

Today, Mr.~Hermes leads GenAI initiatives that shape the future of
agentic business. He helps organizations across sectors build
exceptional AI-powered experiences.

Connect with him at \emph{dan@lexicon.systems}.

gellitos

\section{Acknowledgments}\label{acknowledgments}

To all who trained me on the power of words.

The ChatGPT platform for being both my coauthor and mentor.

The quiet geniuses to which I owe my inspiration and navigation: Nate
Aune, David Alexander, and my father, Michael Hermes.

Everyone with a Reddit or GitHub account.

Reuven Cohen (rUv) who vibe codes in the future.

And to the reader: Thank you for being the kind of mind that asks.

gellitos

\section{Introduction}\label{introduction}

You're holding a different kind of manual.

This isn't a passive read. It's a toolbox. A spellbook. A decoder ring.

We live in an age of conversational intelligence. And yet, the way we
talk to machines is still crude: flat, literal, cautious. That's where
you come in. This book will teach you to prompt like a poet, an
architect, a hacker of thought.

Each chapter is dedicated to one aspect of prompt anatomy: request,
example, format, context, role, and constraint.

When you're finished with this book you'll understand every fundamental
aspect of how to write a good GPT prompt.

As you learn to prompt, it helps to understand two things:

\begin{enumerate}
\def\labelenumi{\arabic{enumi}.}
\tightlist
\item
  What \emph{you're asking for}
\item
  What \emph{the model is doing} to deliver it
\end{enumerate}

They're not the same, but they work together. They begin with your
intention.

``Help me explain this.'' ``Make this more persuasive.'' ``Write it like
I would.'' ``Give me a better way to say it.''

These are actions you're asking for. They define the shape of your need.
They are what this book is built around.

After you ask, the model must do.

\subsection{\texorpdfstring{\textbf{Model Capabilities (or
Features)}}{Model Capabilities (or Features)}}\label{model-capabilities-or-features}

Capabilities (called features in ML-speak) are the things the model is
good at doing, not because it's born clever, but because it's seen
billions of patterns.

Your GPT model can:

\begin{itemize}
\tightlist
\item
  \textbf{Generate} content
\item
  \textbf{Critique} tone or logic
\item
  \textbf{Simulate} roles and voices
\item
  \textbf{Translate} between styles or languages
\item
  \textbf{Reason} through possibilities
\item
  \textbf{Evaluate} options
\item
  \textbf{Plan} structured responses
\item
  \textbf{Reformat} content into new shapes
\item
  \textbf{Reflect} ideas back to you
\end{itemize}

These are the engines under the hood, and new ones arrive with each
release of the model. You don't need to understand them in detail, but
knowing they exist helps you prompt more powerfully.

You say:

``Write me three short replies with a calm, reassuring tone.''

The model:

\begin{itemize}
\item
  Generates
\item
  Simulates tone
\item
  Evaluates for match
\item
  Formats the output
\item
  Adjusts rhythm and length

  ---all automatically.
\end{itemize}

You don't have to name these capabilities. You just have to shape a
prompt that invites them.

The request is the steering wheel. The capability is the engine.

\subsection{\texorpdfstring{\textbf{This Book Is Focused on the
Request}}{This Book Is Focused on the Request}}\label{this-book-is-focused-on-the-request}

We'll empower your requests by exploring the ways to:

\begin{itemize}
\tightlist
\item
  Ask clearly
\item
  Show examples
\item
  Shape the response
\item
  Provide the right background
\item
  Guide the voice
\item
  Set smart boundaries
\end{itemize}

Because once you've made a strong request, the model does the rest.

And in this new world, the ones who shape the prompts shape the future.

\subsection{How to Use This Book}\label{how-to-use-this-book}

Each chapter stands alone and together. Feel free to hop around. Use the
prompts as-is or let them mutate. Combine formats with roles. Mash up
examples and constraints.

Prompts are not just input, they're instructional spells. The more
intentional your language, the more powerful the output.

Let's prompt.

gellitos

\section{Chapter 1: The Request}\label{chapter-1-the-request}

A prompt is not a command. It is a conversation with consequence.

A prompt is the spark you send into the system, to a model like ChatGPT.
It is the first breath of dialogue between you and the intelligence on
the other side of the screen.

You've been prompting your whole life. When you ask your barista: ``Can
you make it stronger today?''

You're prompting when you tell a friend: ``Talk me out of this.''

A prompt is a well-shaped intention. In AI terms, it's a piece of
language crafted to produce a specific result from a language model.

But in \emph{Prompt Power}, we treat it as something deeper:

A prompt is a tool, a mirror, and a spell.

Used well, it doesn't just return information. It returns insight,
possibility, clarity, even peace.

\subsection{\texorpdfstring{\textbf{The Anatomy of a
Prompt}}{The Anatomy of a Prompt}}\label{the-anatomy-of-a-prompt}

A prompt is not a sentence. It's a structure, built from distinct,
modular pieces.

\begin{figure}
\centering
\pandocbounded{\includegraphics[keepaspectratio]{D:/My Documents/Lexicon Systems/AI/Writing/Prompt Power/Anatomy_of_a_Prompt.png}}
\caption{Anatomy\_of\_a\_Prompt}
\end{figure}

In this book, we'll explore each of these pieces in depth:

\begin{enumerate}
\def\labelenumi{\arabic{enumi}.}
\tightlist
\item
  \textbf{Request}
\end{enumerate}

The request is the ask. The intention. The action you want taken. It
lives at the center of the prompt and at the center of the conversation.

\begin{enumerate}
\def\labelenumi{\arabic{enumi}.}
\setcounter{enumi}{1}
\tightlist
\item
  \textbf{Examples}
\end{enumerate}

Models that shape understanding. Examples show the AI what kind of
answer you're after.

\begin{enumerate}
\def\labelenumi{\arabic{enumi}.}
\setcounter{enumi}{2}
\tightlist
\item
  \textbf{Format}
\end{enumerate}

The shape you want the answer to take: list, email, table, monologue,
whatever format the moment calls for.

\begin{enumerate}
\def\labelenumi{\arabic{enumi}.}
\setcounter{enumi}{3}
\tightlist
\item
  \textbf{Context}
\end{enumerate}

The backstory. The setting. The situation. Without context, even a
right-seeming answer can be the wrong move.

\begin{enumerate}
\def\labelenumi{\arabic{enumi}.}
\setcounter{enumi}{4}
\tightlist
\item
  \textbf{Role}
\end{enumerate}

Who is the model supposed to \emph{be} while answering: A teacher? A
CEO? A therapist? A pirate? Give the AI a role.

\begin{enumerate}
\def\labelenumi{\arabic{enumi}.}
\setcounter{enumi}{5}
\tightlist
\item
  \textbf{Constraints}
\end{enumerate}

Word limits, tone, structure, time period, audience. Constraints don't
limit creativity---they focus it.

You won't need all of them every time, but when you know how to use
them---when to add, when to subtract---you can shape conversation with
surgical precision or poetic ambiguity, as the moment requires.

Take for example this book, written with the assistance of ChatGPT. The
tone is gentle and firm all at once, seven chapters of points that are
clear and illustrated with anecdotes. You'll read these stories and get
to know the characters here. Through these concepts you may expand your
creative avenues and career opportunities. You may even enhance your
communication skills. When you're done with the main part of the book
you have an appendix of 500 power prompts to offer you a full depth and
breadth of potential prompt directions.

These are human achievements, manifested through an author/AI
collaboration. One achieved in a collaboration between you and the
author, Dan Hermes, through prompts.

Welcome to a world where AIs will teach humans how to interact with AIs.

Can you begin to see how powerful all this can be?

The final page of this book provides you with the prompt the author used
to write it so you may have insight into how this book was created. In
the meantime, consider most of what you read here to be living examples
of the power of the prompt:

\textbf{Prompt Power.}

And with that, the title of this book, let us begin at the beginning:
with the request.

\subsection{\texorpdfstring{\textbf{Why We Start with the
Request}}{Why We Start with the Request}}\label{why-we-start-with-the-request}

The center of the prompt is the desire.

Traditionally, the ``request'' is called the instruction --- a term
borrowed from programming, technical documentation, and command-line
interface culture. But this book is not about issuing orders.

It's about shaping conversations, making meaning in partnership with a
powerful mind. It's about asking questions that matter, in ways that
land.

We don't use the word ``instruction'' in this book, a term from the era
of programmatic rules. We use request, because that's what it is:

\begin{itemize}
\tightlist
\item
  A request for clarity
\item
  A request for story
\item
  A request for help
\item
  A request to be heard
\end{itemize}

And the way you request something shapes what is possible in return.

\subsection{\texorpdfstring{\textbf{What a Request Can
Be}}{What a Request Can Be}}\label{what-a-request-can-be}

A request can be bold or quiet. It can be precise or abstract. It can
ask for an output\ldots{} or an insight.

Here are a few types of requests you might make:

\textbf{Business Requests}

``Write a short email to follow up after a sales call.'' ``Summarize
these meeting notes into action items.'' ``Generate 5 taglines for a
startup that builds solar-powered fridges.''

\textbf{Creative Requests}

``Turn this short story into a screenplay opening.'' ``Write a breakup
letter in the voice of a 19th-century vampire.'' ``Help me name a
fantasy city built on the back of a whale.''

\textbf{Emotional or Relational Requests}

``Help me explain to my partner why I'm feeling distant lately.''
``Write a birthday message to my father that feels real, not robotic.''
``Give me words to support a friend who just lost their job.''

\textbf{Reflective or Personal Requests}

``Help me list the values I actually live by, not just the ones I say I
believe in.'' ``Write a note from my future self, encouraging me to keep
going.'' ``What am I really afraid of in this situation?''

\textbf{Organizational or Strategic Requests}

``Help me turn this chaotic brainstorm into a structured project plan.''
``Translate this vision statement into something my team will actually
care about.'' ``Find the gap in this business model.''

You've seen the breadth of what you can ask (by domain). Now let's look
at why we ask at all. Behind every prompt is something older than code.
A reason. A hunger. A spark. Prompts don't begin with the keyboard. They
begin with a desire: I want to know\ldots{} I want to say\ldots{} I want
to try\ldots{}

Let's walk through the five deeper impulses that drive people to prompt
in the first place.

\subsection{\texorpdfstring{\textbf{Why We Prompt: Five Human
Needs}}{Why We Prompt: Five Human Needs}}\label{why-we-prompt-five-human-needs}

Behind every prompt is a human reason for asking. These aren't
categories of content. They're modes of intent. Doorways into the act of
requesting itself. These are some reasons we need to prompt:

\subsubsection{\texorpdfstring{1. \textbf{For Clarity in Professional
Life}}{1. For Clarity in Professional Life}}\label{for-clarity-in-professional-life}

We seek sharpness. We ask to \emph{see} the idea before it's seen by
others.

``Generate an opening paragraph for a proposal that conveys urgency,
without sounding desperate.'' The balance of tone starts with the
balance of intent.

``Summarize this in three concise bullet points for a team that's been
working on it for months(background attached).'' For when they already
know the backstory---give them the path forward.

``Here's the situation: {[}brief context{]}. I need a {[}type of
content{]} that does {[}goal{]}.'' Use this to stop circling and start
building.

\subsubsection{\texorpdfstring{2. \textbf{For Connection in Personal
Moments}}{2. For Connection in Personal Moments}}\label{for-connection-in-personal-moments}

We want help finding the words that make meaning between people.

``Help me write a message that asks for {[}emotional need{]} without
sounding needy or confrontational.'' Most emotional clarity is born from
tone, not grammar.

``Give me three gentle ways to say: `I need space.'\,'' A direct request
doesn't have to be a sharp one.

``Here's what I wish I could say out loud. Turn this into something I
could actually send.'' The truth wants a bridge, not a bullhorn.

\subsubsection{\texorpdfstring{3. \textbf{For Creativity in
Expression}}{3. For Creativity in Expression}}\label{for-creativity-in-expression}

We reach for a spark, a strange turn, a new way to say the old thing.

``Act as my creative collaborator. Help me ask for feedback that invites
suggestions, not criticism.'' How you ask shapes how others respond.

``Turn this vague idea into a concrete creative brief I can send to a
designer.'' Dreams need format to survive outside your head.

\subsubsection{\texorpdfstring{4. \textbf{For Collaboration in
Building}}{4. For Collaboration in Building}}\label{for-collaboration-in-building}

We prompt to make together---plans, drafts, ideas, decisions.

``Write a polite but confident request for collaboration that respects
the other person's time.'' You're not begging, you're inviting.

\subsubsection{\texorpdfstring{5. \textbf{For Guidance When We're
Lost}}{5. For Guidance When We're Lost}}\label{for-guidance-when-were-lost}

We turn to the page with no idea what to say yet. And it listens anyway:

``Help me figure out what I'm really asking for.'' The most powerful
requests often begin with a question about the request itself.

Perhaps you can see why discussion of human needs supersede technical or
business discussions in this Prompt 101 primer. The power of AI is not
in its business acumen but in its understanding of the world and the
life of humanity within it. Upon that foundation is built disciplines,
industries, families, and art. When we prompt we draw upon that vast
understanding to help meet our own human needs.

\emph{Prompt Power} begins with the shape of the ask. The clearer your
need, the clearer the door. A request without shape is a whisper into
fog. If the request is weak, the format won't save it. The role won't
carry it. The tone will be smoke without flame. But if the request is
alive --- clear, honest, well-shaped --- everything else becomes
refinement.

The rest of this book will help you to refine. But this chapter, this
first turn of the key, is about learning how to ask.

\subsection{\texorpdfstring{\textbf{Elena's New
Launch}}{Elena's New Launch}}\label{elenas-new-launch}

Elena worked in product marketing for a small edtech startup. That
morning the team emailed her: ``We need skybeam taglines for the new
quiz app. Something persuasive. Something that works.''

She opened ChatGPT, stared at the blinking cursor, and typed: ``Write
some persuasive content for our new student quiz app to raise
visibility''

The response wasn't wrong , but it wasn't taglines. It rambled about
benefits. Suggested social posts.

She sighed. She hadn't made the wrong request. She'd made a vague one.
So she specified her ask:

``Generate five persuasive taglines for an adaptive quiz app for
students with learning differences. The tone should be clear, warm, and
trustworthy.''

This time: ``Learning, Your Way.'' ``Support That Adapts to You.''
``Smart Tools, Gentle Guidance.'' ``Study Without Struggle.''
``Confidence Begins Here.''

Better, and now she had something to work with. It wasn't a smarter
model. It was a smarter request.

\subsection{\texorpdfstring{\textbf{What Was Elena Really Asking
For?}}{What Was Elena Really Asking For?}}\label{what-was-elena-really-asking-for}

Elena didn't get a better result because she used fancier language. She
got one because, in her second prompt, she asked for something specific.
She didn't just say ``Write some content,'' she said:

``Generate five taglines\ldots''

That's a request.

It's the part of the prompt where you say what you want done. Not what
it's about. Not what voice to use. Not who the audience is. Those are
all important, but they orbit around the request, which is the core
action.

This book will show you hundreds of ways to prompt, but they all begin
with the verb---the \emph{thing you want done}. Here are a few:

Request to \textbf{Create}: ``Write an introduction\ldots'' ``Generate 3
ideas\ldots'' ``Compose a short message\ldots'' ``Design a sample
outline\ldots''

Request to \textbf{Revise}: ``Make this sound more confident\ldots''
``Shorten this paragraph\ldots'' ``Improve the flow without changing the
meaning\ldots''

Request to \textbf{Translate or Transform}: ``Explain this like I'm
five.'' ``Turn this chart into a short paragraph.'' ``Rephrase this with
more empathy.''

Request to \textbf{Reflect} or \textbf{Analyze}: ``Tell me what's
missing from this list.'' ``What's unclear about this?'' ``How could
this land badly?''

Request to \textbf{Guide}: ``Help me organize this.'' ``Give me 3 next
steps.'' ``How should I start?''

Request to \textbf{Discover}: ``What am I really asking here?'' ``List
possible intentions behind this request.'' ``What questions should I be
asking instead?''

These are the actions---the verbs---you're putting into the model's
hands. That's what request means. When you prompt, always ask yourself:
What do I want it to do? Am I asking it to create, refine, clarify,
challenge, or translate?

If that part isn't clear, no amount of tone or detail will save the
rest.

Let's see how a request can work in a team.

\subsection{\texorpdfstring{\textbf{Elena and the Meeting That Didn't
Ask}}{Elena and the Meeting That Didn't Ask}}\label{elena-and-the-meeting-that-didnt-ask}

The marketing war room was full.

Eight team members. Two open laptops. One flickering projector. And zero
alignment.

They were prepping for a partner launch---a joint campaign with a big
national nonprofit. The stakes were high. So were the expectations. And
everyone had ideas:

``We need a statement of values,'' said one. ``An origin story!''
exclaimed another. ``A headline that makes people care,'' offered
someone else. It was clear they didn't know what they were asking for,
and it wasn't because they didn't care. It was because no one had shaped
the ask. After twenty minutes of circling, Elena---leaning back in her
chair, arms folded---said quietly:

``Can we pause?''

The room stilled. She opened her laptop. ``Let's try something. I'm
going to prompt ChatGPT with a real request. Just one. Let's see what we
get.''

She typed: ``Write a short, mission-driven paragraph introducing a
partnership between an edtech startup and a national nonprofit that
supports inclusive learning. The tone should be confident but warm. This
is for a landing page seen by parents and educators.''

A moment later, the model returned: ``Together, we're building a future
where every student, regardless of ability, has access to learning tools
that adapt to their needs. Our partnership unites innovation with
inclusion, creating real change where it matters most: in the
classroom.''

The room was quiet. Then someone said: ``Okay. That's something.''

Now they had a starting point. Something to critique. A shape. A
direction. A launchpad. The team had been talking around the work. Elena
had prompted into it.

Not because she was the best writer in the room, but because she'd made
a clear request.

\subsection{\texorpdfstring{\textbf{When a Group Talks but No One
Asks}}{When a Group Talks but No One Asks}}\label{when-a-group-talks-but-no-one-asks}

Teams often talk in ideas. A request turns talk into action. You don't
have to be the loudest in the room to make the smartest ask.

A good request isn't about sounding clever. It's about asking for
something real. This is what often happens in teams: everyone brings
ideas, intentions, opinions, but no one brings a request. And without a
clear request, there is no clear response. What we saw in Elena's
meeting wasn't a failure of creativity. It was a failure to form the
ask. They needed:

\begin{itemize}
\tightlist
\item
  A paragraph
\item
  For a landing page
\item
  About a partnership
\item
  With a tone that matched the moment
\end{itemize}

But until Elena named it, no one said that out loud.

Notice that Elena didn't ask the model to ``write something great.'' She
asked it to do something small, specific, and actionable. It wasn't
about producing a final version, it was about creating a starting point.
That's what a good request does:

It gives you something to push off of. It turns discussion into
material. It says: ``Do this, so we can respond to it.''

\subsection{\texorpdfstring{\textbf{This Is Why We Start with
Request}}{This Is Why We Start with Request}}\label{this-is-why-we-start-with-request}

You could spend hours fine-tuning tone, audience, formatting, but if the
action you're requesting isn't clear, you're just spinning words. In the
chapters that follow, we'll build on this. We'll explore how examples,
format, context, and more shape the conversation. But every strong
prompt begins with this question:

``What do I want it to do?''

That's the heartbeat of the request. And if that's the only thing you
master, you'll already be miles ahead of most.

The request isn't just the center of the prompt. It's the gateway to
discovering what AI can do for you:

\begin{itemize}
\tightlist
\item
  To reframe a problem
\item
  To diagnose what's missing
\item
  To suggest next steps
\item
  To rewrite, summarize, challenge, restructure, condense, expand, or
  reposition your work
\item
  To play a role, take a side, show empathy, or give hard feedback
\end{itemize}

You can ask it to explain something five different ways. Or five times
better. Or as if your boss needs it in two minutes and has never read a
brief in her life. And if you don't ask, it won't offer.

A vague ask will yield a vague response. Requests are not questions:
they are intentions in disguise.

The clearer your need, the clearer the answer.

\subsubsection{\texorpdfstring{\textbf{Seeing the
Menu}}{Seeing the Menu}}\label{seeing-the-menu}

Before you can become a powerful prompt crafter, you need to expand your
idea of what's askable.

You don't need technical skill. You don't need a programming background.
You don't need to sound impressive.

You just need to know what you can ask for:

\begin{itemize}
\tightlist
\item
  Action
\item
  Advice
\item
  Strategy
\item
  Persuasion
\item
  Revision
\item
  Reinvention
\item
  Clarity
\item
  Creativity
\item
  Compassion
\end{itemize}

Let's examine each of these.

\subsubsection{\texorpdfstring{\textbf{Action}}{Action}}\label{action}

Sometimes, you just need it to do something. Not imagine. Not explain.
Just do.

``Write a subject line for this email.'' ``Turn this outline into a blog
post.'' ``List five ways to introduce a speaker.''

These are requests that create. They are the digital equivalent of
saying, \emph{``Can you do this real quick?''}

When you don't know where to start, ask it to act. Not because it's
perfect. But because it's moving.

\subsubsection{\texorpdfstring{\textbf{Advice}}{Advice}}\label{advice}

You can ask for guidance. Gently. Boldly. Anonymously.

``What's the best way to prepare for a raise conversation?'' ``How do I
announce I'm leaving a job without burning bridges?'' ``What are common
mistakes when launching a new feature?''

It doesn't mean you take the answer as gospel, but it gives you
something to push against. Something to consider. Advice is a request
for wisdom, not facts. It turns the model from a library into a
conversation partner.

And sometimes, hearing advice out loud, even if it's not quite right,
can help you believe what you already knew.

\subsubsection{\texorpdfstring{\textbf{Strategy}}{Strategy}}\label{strategy}

You can ask for thinking. Structured, layered, actionable thinking.

``Give me a 3-phase rollout plan for this campaign.'' ``What's a smart
way to increase engagement among lapsed users?'' ``Compare two
approaches to pricing this product.''

Strategy prompts don't just ask what to do, they ask how to think about
it. This is where the model shines: combining known patterns, surfacing
assumptions, suggesting frameworks. You don't have to take its plan
whole. But you might steal its bones.

\subsubsection{\texorpdfstring{\textbf{Persuasion}}{Persuasion}}\label{persuasion}

You can ask it to help you win hearts, shift minds, or open doors.

``Help me write a message that gets someone excited about our product.''
``How can I frame this feature as a solution, not a detail?'' ``Make
this offer feel more urgent without sounding pushy.'' ``Give me three
angles to pitch this to a skeptical CFO.''

Persuasion prompts are about alignment---not trickery. You're not asking
it to manipulate. You're asking it to help connect value to need. And
sometimes, all it takes is a better angle. A new lead-in. A clearer path
between what you offer and what they care about.

That's persuasion. And yes, you can ask for it.

\subsubsection{\texorpdfstring{\textbf{Revision}}{Revision}}\label{revision}

You can ask it to make what you've written better.

``Shorten this without losing meaning.'' ``Make this sound more
confident.'' ``Fix the grammar and punctuation in this email.'' ``Change
the tone to be more compassionate.''

Revision isn't cheating, it's collaboration. You bring the thought. It
helps shape the delivery.

Sometimes you don't need a new idea. You just need the one you
had---refined.

\subsubsection{\texorpdfstring{\textbf{Reinvention}}{Reinvention}}\label{reinvention}

You can ask it to change form completely.

``Turn this report into a slide deck.'' ``Rewrite this memo as a press
release.'' ``Take this outline and turn it into a podcast script.''

This is powerful. It's not just wordsmithing---it's format-shifting. A
reinvention prompt helps you take what you already have, and see how
else it might live. This is especially helpful when you have something
half-done but need to adapt it to a new context.

\subsubsection{\texorpdfstring{\textbf{Clarity}}{Clarity}}\label{clarity}

You can ask it to make something more accessible and understandable.

``Explain this in simple terms.'' ``What parts of this message might
confuse people?'' ``Summarize this in 3 key takeaways.''

Clarity is about understanding. And sometimes the most helpful thing you
can do is say to the model: ``Help me see this better.''

Clarity prompts are powerful for revising things you didn't write --- a
colleague's doc, a legacy policy, a block of messy notes.

You don't need brilliance. You need visibility.

\subsubsection{\texorpdfstring{\textbf{Creativity}}{Creativity}}\label{creativity}

You can ask it to play.

``Give me 5 names for a mysterious candle scent.'' ``Invent a holiday
for introverts.'' ``What would a brand voice sound like if it were a
forest?''

Creativity doesn't mean random, it means unusual, expressive,
unexpected. You can ask for metaphors. Imagery. Rhymes. Characters.
Contrasts. The best creative prompts give a shape and invite surprise.

This isn't about getting the final result. It's about generating raw
material you can remix, reject, or rearrange.

\subsubsection{\texorpdfstring{\textbf{Compassion}}{Compassion}}\label{compassion}

You can ask it to help you say hard things softly.

``Write a message to a client whose expectations we missed.'' ``Give me
a way to apologize that sounds real.'' ``How do I respond to a teammate
who's grieving?''

This is one of the most underused kinds of prompting, but one of the
most powerful. You don't need a robot to feel, but you might need help
finding the language for feeling. Compassion prompts are often the most
personal, but they also show up at work more than you'd expect. Because
so much of our work is with people.

The model won't always get it right, but it won't get anything unless
you ask. And now you know: you can.

\subsection{\texorpdfstring{\textbf{Closing: The Core of the
Conversation}}{Closing: The Core of the Conversation}}\label{closing-the-core-of-the-conversation}

Every prompt you write begins with a request. Not just because it's the
first thing you type --- but because it's the center of the intention.
What do you want the model to do? That's the heartbeat.

The model doesn't know. It waits. It listens. And the request is what
tells it:

``Move. Make. Clarify. Create. Soften. Sell. Summarize. Solve.''

This is the most human part of prompting --- the desire behind the
words.

The rest of this book walks through the tools that make a request work
better:

\begin{itemize}
\tightlist
\item
  Examples show the model what success looks like.
\item
  Format tells it how to shape the response.
\item
  Context gives it the backstory it needs to respond intelligently.
\item
  Role asks it to step into a persona that fits your situation.
\item
  Constraints keep things focused, polished, and useful.
\end{itemize}

Each one adds clarity to the ask. Each one helps the model not just
respond, but respond well. But make no mistake:

Without a clear request, there's nothing for the rest to serve. That is
why we start here.

Because everything that comes next --- every clever structure, every
precise tone, every impressive outcome ---

only matters if you've first asked for something worth answering.

\subsection{\texorpdfstring{\textbf{The Model Needs a
Model}}{The Model Needs a Model}}\label{the-model-needs-a-model}

So you've made your request. It's clear. It's honest. It's alive. But
the model still has a question:

``What does a good answer look like?''

This is where many prompts fall flat. Not because the ask is weak but
because the model has nothing to model.

This is where you stop telling and start showing.

In the next chapter we'll explore how to provide examples so the model
isn't just guessing at what you want, but drawing from patterns you've
given it. Because some of the best prompts don't just ask for something.

They show what success looks like.

gellitos

\section{Chapter 2: Give Examples}\label{chapter-2-give-examples}

If the request is the command, the example is the lesson. Nothing guides
an AI more powerfully than a demonstration. This chapter shows you how
to build your prompts around showing, not just telling.

We call it Machine Learning (ML) because AIs identify and learn from
patterns, which we call training. That makes prompts work particularly
well when they include examples.

Examples don't constrain the model---they give it guideposts. They tune
the output to match your voice and your rhythm. They say: ``Do it like
this.'' And the model listens.

\subsection{How the Model Learns From What You
Show}\label{how-the-model-learns-from-what-you-show}

You've asked for something. The model is listening. But it still doesn't
know what you mean.

You said ``make it friendly.'' You said ``a sharp summary.'' You said
``help me pitch this.''

But what does ``friendly'' sound like? What does a ``sharp summary''
include? What kind of pitch do \emph{you} think works?

This is where many prompts fail---not in the asking, but in assuming
that the model already knows what success looks like.

And that's where \textbf{examples} come in.

\subsection{Tell Me vs.~Like This}\label{tell-me-vs.-like-this}

An example says:

I don't just want a good answer. I want a good answer that looks like
\textbf{\emph{this}}.

It might be:

\begin{itemize}
\tightlist
\item
  A sentence you've already written
\item
  A fake example with the structure you want
\item
  A tone sample, a snippet, a prototype
\item
  Even your improved rewrite of what the model just gave you
\end{itemize}

You're showing it a pattern. And language models thrive on pattern.

You don't have to know exactly what you want. But if you can show
something close---something half-right, something shaped---you're giving
the model a mirror to reflect from.

A sample. A frame. A hint. That's all it needs to begin improving.

\subsection{\texorpdfstring{\textbf{The
Apology}}{The Apology}}\label{the-apology}

It was a small mistake --- a broken link in a campaign email. But the
customer it affected was a large one, and their reply came in hot:
``This isn't the first time. We expect better.''

The team gathered in a virtual huddle. Heads down. Tones clipped. Elena
was there again --- tasked with drafting the reply. They started with
the basics:

``Say we're sorry.'' ``Make it professional.'' ``Keep it short.''
``Don't over-apologize. But don't underdo it either.''

Elena typed a prompt into ChatGPT:

``Write a short professional email apologizing for a broken link in a
customer campaign. Be warm but not overly apologetic. Keep it respectful
and concise.''

The response came back fast, and it was\ldots{} fine:

``We regret the inconvenience caused by the broken link in our recent
campaign. We've addressed the issue and are taking steps to prevent it
from happening again. Thank you for your understanding.''

Nobody hated it, but nobody wanted to send it either. ``Still feels
stiff,'' someone said. ``Yeah, it doesn't sound like us,'' said another.
Elena sighed. She had asked for the right thing, but she hadn't shown it
yet.

So she copied an email they'd written months ago, sent to a partner when
something similar had gone sideways. It had the tone they wanted:
Direct. Thoughtful. Not too polished. Just real. She added it to the
prompt:

``Here's an example of our voice in a similar situation:

`Thanks for flagging that. We definitely missed the mark, and we
appreciate your patience. We're correcting it now and we'll be following
up with an updated version shortly. Let us know if you spot anything
else---we're watching it closely too.'

Use this as a tone reference and rewrite the apology email to the
customer who received the broken link.''

The model tried again:

`We really appreciate you pointing this out. The link issue slipped
through and we've already fixed it. We're keeping a close eye to make
sure it doesn't happen again. Thanks for your patience --- and for
holding us to a higher standard.'

This time, the reply felt like it came from \emph{them}. The team
nodded. ``Yep. That's it.''

Send.

\subsection{\texorpdfstring{\textbf{Takeaways}}{Takeaways}}\label{takeaways}

\begin{itemize}
\tightlist
\item
  Even a perfect request may fail if the model doesn't know what
  ``good'' looks like.
\item
  You don't have to provide a perfect sample---just a clear one.
\item
  Examples aren't cheating, they're collaborating.
\end{itemize}

Elena didn't change the request. She didn't rephrase it. She didn't
clarify the audience.

She showed an example.

The model didn't have to guess what she meant by ``professional'' or
``warm.'' It saw it. It heard the voice she wanted. It mapped her tone,
her phrasing, her rhythm. Language models can mimic almost any writing
style --- if you give them something to mimic.

That's what an example does: It creates a pattern.

You can drop in a sample message. A sample phrase. Even just a list of
adjectives you like. And the model will say:

``Ah. You want something like this. Got it.''

\subsection{\texorpdfstring{\textbf{Example
Types}}{Example Types}}\label{example-types}

Let's walk through a few common types of examples you can include in
your prompt---and what they're best for.

\subsubsection{\texorpdfstring{\textbf{1. Tone
Example}}{1. Tone Example}}\label{tone-example}

\emph{A snippet that captures the voice you want}

Use this when you want the model to match a vibe---playful, confident,
bold, academic, whatever.

\textbf{Prompt:} ``Here's the kind of tone we usually use: We know
security's complicated. That's why we make it simple. We handle the hard
parts so you don't have to.

Use this tone to write a landing page for our new compliance product.''

Tone examples are often short, just 1--3 sentences, but powerful. They
tell the model how to sound.

\subsubsection{\texorpdfstring{\textbf{2. Structural
Example}}{2. Structural Example}}\label{structural-example}

\emph{A sample that shows the format or flow you want}

Use this when you want the model to follow a pattern---like a list, a
framework, or a style of explanation.

\textbf{Prompt:} ``Write a thought leadership post in the same structure
as this one:

\begin{itemize}
\tightlist
\item
  Start with a question.
\item
  Share a surprising insight.
\item
  Use a metaphor.
\item
  End with an invitation.
\end{itemize}

Use this structure to write a post about burnout in startup founders.''

Structure examples give the model a skeleton to build upon. They're
especially useful for content creation.

\subsubsection{\texorpdfstring{\textbf{3. Before and
After}}{3. Before and After}}\label{before-and-after}

\emph{Show what you don't want---and then what you do}

This is especially helpful for revising or rewriting prompts.

\textbf{Prompt:} ``Here's the version we don't want: We are reaching out
to you in regards to the following issue\ldots{}

And here's what we prefer: Just wanted to flag something quickly---we
noticed a small issue and we're on it.

Rewrite this message to match the second style.''

This helps the model \emph{course correct} toward your ideal.

\subsubsection{\texorpdfstring{\textbf{4. Self-Correction
Example}}{4. Self-Correction Example}}\label{self-correction-example}

\emph{Take what the model gave you---and rewrite it better}

Sometimes, the best example comes from your own edits.

\textbf{Prompt:} ``You wrote this: We understand your frustration and
are working to address the issue.

I rewrote it like this: Thanks for calling this out. We're on it---and
we'll follow up shortly with an update.

Use my version as a tone reference for future responses.''

This teaches the model by showing how your version differs from its
guess.

\subsubsection{\texorpdfstring{\textbf{5. Imaginary
Sample}}{5. Imaginary Sample}}\label{imaginary-sample}

\emph{Make up an example to guide the model}

You don't always need a real-world quote. You can write a made-up one,
just to create a pattern.

\textbf{Prompt:} ``Write a customer testimonial that sounds like this:
`The interface is gorgeous, but what really sold me was the support
team. They're like having tech whisperers on call.'

Generate 5 more in this tone.''

\subsubsection{\texorpdfstring{\textbf{6. Instructional
Examples}}{6. Instructional Examples}}\label{instructional-examples}

\emph{A sample paired with an explanation of why it works}

This one is great for onboarding your team's voice or rules.

``When we respond to customer complaints, we use this format:

\begin{enumerate}
\def\labelenumi{\arabic{enumi}.}
\tightlist
\item
  Acknowledge the issue
\item
  Take responsibility if it's ours
\item
  Outline what we're doing
\item
  Thank them for their patience
\end{enumerate}

For example: `You're right---we missed that. It's been fixed, and we've
updated the docs to make sure it doesn't happen again. Appreciate you
flagging it.'

Use this format to respond to a user who found a bug in the mobile
app.''

You're not just giving an example---you're teaching with it.

\subsubsection{\texorpdfstring{\textbf{7. Bad
Examples}}{7. Bad Examples}}\label{bad-examples}

\emph{Provide a bad example and ask the model to avoid it}

This is especially helpful when tone is critical.

``Don't write like this:

We apologize for the inconvenience and regret any negative impact this
may have caused.

It's too cold and robotic. Make it sound more human.''

Bad examples set a boundary: ``Anywhere but there.''

\subsubsection{\texorpdfstring{\textbf{8. Persona
Example}}{8. Persona Example}}\label{persona-example}

\emph{Examples generated by a specific person or group}

``Our CEO tends to write like this:

`We're learning fast, and we'll get it right. That's a promise, not a
platitude.'

Use this voice to respond to a critical investor email.''

These examples help the model match point of view.

Those are some of the most common example prompts: Tone, Structural,
Before and After, Self-correction, Imaginary, Instructional, Bad, and
Persona. More examples can be better than fewer but the important thing
is quality.

\subsection{\texorpdfstring{\textbf{You Just Need One That
Lands}}{You Just Need One That Lands}}\label{you-just-need-one-that-lands}

You don't need to give five examples. You don't need a whole doc.
Because once the model sees how you want it to sound, it doesn't have to
guess. Sometimes one Is enough, but five paints a picture. When you
include one example in a prompt, you give the model a signal. When you
include several, you give it a pattern. That's the difference between
guessing the vibe --- and locking it in.

Multiple examples help the model:

\begin{itemize}
\item
  Spot consistency
\item
  Weigh what matters most
\item
  Avoid outliers
\item
  And generate results that feel intentional, not random
\end{itemize}

Multiple examples like this:

``Here are five taglines from past campaigns we liked:

\begin{itemize}
\tightlist
\item
  Clear Power for Complex Systems
\item
  Control Without the Chaos
\item
  Every Detail, Under Control
\item
  Power, Made Predictable
\item
  Nothing Slips Through
\end{itemize}

Generate five new taglines for our next product launch. Keep the tone
similar---crisp, direct, quietly confident.''

That's not just prompting. That's training. You're not hoping it
understands your taste --- you're \emph{showing} it.

Multiple examples like this: (get it?)

``Our support team tends to sound like this:

\begin{itemize}
\tightlist
\item
  `Totally fair point. Let me dig into that and I'll follow up by EOD.'
\item
  `Appreciate your patience---this one's on us. Fix incoming.'
\item
  `Good catch! Flagging that for the product team now.'
\end{itemize}

Using this tone, respond to a user who reported a broken button.''

This doesn't just produce a usable response. It helps your AI co-worker
speak in your voice.

If you give five examples that sound wildly different, the model won't
know which way to steer. But if they share tone, structure, or rhythm,
it can match the pattern beautifully. More cohesive examples = more
certainty. Especially when you're going for tone matching or voice
generation at scale.

Here's Elena mentoring a co-worker on how to do it.

\subsection{\texorpdfstring{\textbf{Elena and the Junior
Marketer}}{Elena and the Junior Marketer}}\label{elena-and-the-junior-marketer}

The meeting had ended. The email was sent. The tension in the
room---dissolved. But Elena noticed Jake lingering.

He was new. Smart. Eager. And clearly frustrated. ``You made that look
easy,'' he said, ``But I asked ChatGPT for the exact same thing earlier
and it gave me junk.''

She smiled, ``Did you show it anything?''

``Well\ldots{} no. I told it what I wanted.''

``Right,'' she said. ``But sometimes you need to show, not just tell.''

They sat down in a side room, two laptops open, coffee between them.
``What were you trying to write?'' she asked.

A follow-up message to partners who came to our demo. I wanted it to be
short and appreciative, but with a clear next step.

``Got the prompt?'', she asked. He pulled it up:

``Write a short follow-up email thanking partners for attending our
product demo and encouraging them to sign up for a trial.''

It wasn't a bad prompt. The AI's reply was something like:

Thank you for attending our recent demo. We hope you found it
informative. If you haven't already, we encourage you to sign up for a
trial today.

It was the kind of email you delete before the second sentence. ``Now
watch this,'' Elena said. She dug into a folder and copied in three
messages they'd sent after past events. They had voice. Energy.
Personality.

She pasted them under the prompt: ``Here are three examples of the tone
we usually use:

\emph{`Thanks again for coming to the demo---great questions, by the
way. If you're curious what it's like to actually use the tool, hit the
link below and try it free for 14 days.'}

\emph{`Loved having you join us. This tech only makes sense when you get
your hands on it---grab a free trial and see what we mean.'}

\emph{`Appreciate you showing up! If you're ready to test-drive what we
showed off, here's your chance---free for two weeks, no strings.'}

Write a new follow-up email using this tone and structure.''

The model responded with something Jake hadn't expected:

Thanks again for joining our demo. Want to try it for yourself? You can
start a free trial today---no setup, no commitment. We'd love to hear
what you think.

Jake stared, ``That's actually\ldots{} kind of great.'' ``Right?'',
Elena said, ``Because we showed it what great sounds like.''

\subsection{\texorpdfstring{\textbf{Takeaways}}{Takeaways}}\label{takeaways-1}

\begin{itemize}
\tightlist
\item
  Examples don't just improve the output---they teach the model how you
  think.
\item
  If your prompt is falling flat, don't just say more. Show more.
\item
  One good example is a flashlight on the subject. Three is a
  floodlight.
\end{itemize}

Jake had a decent request. The model responded with a passable email. It
sounded like\ldots{} everyone else. Because one of the most
underappreciated truths in prompt crafting is this:

\emph{The model doesn't default to your voice. It defaults to the most
common voice.}

And most common isn't what your brand wants. It's not what your company
rewards. It's not what makes anyone click.

That's why multiple examples are so powerful. Especially when:

\begin{itemize}
\tightlist
\item
  You're writing in a distinctive voice
\item
  You're producing repeatable content at scale
\item
  You want tone consistency across a team
\item
  You're handing off drafts to others (or to the model itself)
\item
  You want the model to learn, not just respond
\end{itemize}

This is how voice gets trained --- Not with rules, but with recognizable
repetition.

Elena didn't just prompt better. She showed Jake that a good prompt
isn't only about asking, it's about modeling, and that's what examples
are at their best: Small models of what good looks like.

Just as Elena provided a great example for Jake, a coach must do the
same for their players.

\subsection{\texorpdfstring{\textbf{A Tale of Two
Coaches}}{A Tale of Two Coaches}}\label{a-tale-of-two-coaches}

Both teams ran the same drill that day. Same field. Same sweat. Same
burning sun.

On one sideline, a coach barked:

``Run it again!''

And the player did. Fast. Wild. Heart in it, but guessing.

``No!'' the coach snapped. ``That's not what I meant!''

``Then what \emph{did} you mean?''

The coach waved his hand and said it louder. Said it harder.

``Run it again! This time get it right!!''

So the player kept running. Kept missing. And the drill dissolved.

On another sideline, the coach of the winning team watched. When her
player asked:

``How do you want me to run it?''

She didn't raise her voice. She raised her hand and traced the shape of
the movement in the air.

The player smiled, nodded once, and moved like he'd already seen the
success in a dream.

And the drill--- the same one as on the other field--- became poetry in
cleats.

The scoreboard wouldn't show it yet. But that moment was the win.

\subsection{Post-game Analysis}\label{post-game-analysis}

A coach yelling ``do it better'' is no different than a boss saying
``write me something great,'' or a prompt that says ``Do this.'' Without
example, there is no aim. Just volume.

Without examples, there's no shared reality. We speak the same language,
but imagine different outcomes. And that's where things fall apart.

The coach who wins isn't the one who yells louder. It's the one who
gives their desire a shape.

\subsection{\texorpdfstring{\textbf{Reflection}}{Reflection}}\label{reflection}

You can write the best request in the world--- clear, honest, focused---
and still get something that feels \emph{off.}

Because the model doesn't just need to know what you want. It needs to
know \emph{what good looks like.}

That's what examples are for.

They're not instructions. They're impressions. Mirrors that let the
model see what you see.

You've seen the power of:

\begin{itemize}
\tightlist
\item
  \textbf{Tone Examples} --- A few lines of voice can shape a whole
  paragraph
\item
  \textbf{Structural Examples} --- Give it a frame, and it fills it well
\item
  \textbf{Before and After} --- Teach by contrast
\item
  \textbf{Self-Correction} --- Edit what it gave you, then say: ``Do it
  like this.''
\item
  \textbf{Imaginary Examples} --- You don't need real data to make a
  useful sample
\item
  \textbf{Instructional Examples} --- Spell out the format and show it.
\item
  \textbf{Bad Examples} --- What not to do.
\item
  \textbf{Persona Examples} --- Show it how \emph{that person} might say
  it
\end{itemize}

You don't always need five. Sometimes one good example is enough. But
where tone and trust matter--- more quality examples mean less risk.

Because the model doesn't guess style. It copies it. And it can only
copy what you give it.

\subsection{\texorpdfstring{\textbf{Shape the
Shape}}{Shape the Shape}}\label{shape-the-shape}

You've made your request. You've shown what good sounds like. But the
model still wants to know:

``How long should this be?'' ``Should it be a list? A letter? A chart?''
``Where should I stop?''

That's where format comes in. It doesn't change \emph{what} you're
asking for. It changes how the answer appears.

And that can make all the difference.

Ready to shape the shape?

gellitos

\section{Chapter 3: How to Format Your
Output}\label{chapter-3-how-to-format-your-output}

It's not just about content---it's about \emph{shape}. Format is the
shape of a prompt's output.

This chapter explores the power of formatting prompts to direct layout,
sequence, rhythm, and even emotional resonance. From slides to scripts,
tables to outlines---these are the forms that comprise information.

Let us turn now to a format of yore in a tale by Jorge Luis Borges.

\subsection{\texorpdfstring{\textbf{The King and the
Mapmaker}}{The King and the Mapmaker}}\label{the-king-and-the-mapmaker}

The king summoned the mapmaker to the great hall. He was tired of being
surprised by border raids and unruly rivers.

He pointed to a blank table and said: ``Make me a map of the kingdom.''

The mapmaker bowed: ``What kind of map, Your Majesty?''

``The whole thing,'' the king said, waving. ``All of it.''

So the mapmaker set upon the charge with obsessive devotion. He measured
every stream, every barn, every alley. He drew every stone wall, every
goat path, every elevation. In its sublime fastidiousness, the map, when
finished, was surprisingly large.

It was, in fact, the size of the kingdom itself.

The king stared at it, gobsmacked: ``Mapmaker, what have you done? I
can't use this. I can't carry this. No one can! What good is a map you
can't hold in your hands?''

The mapmaker blinked, and reminded the king of his request:

\emph{``You said the whole kingdom.''}

The mapmaker lived out his days in the dungeon. The king remained lost
in his own kingdom.

\subsection{Here Be Dragons}\label{here-be-dragons}

The King had content---too much of it. He had land, lords,
livery\ldots{} even loyalty.

But what he lacked was form.

He didn't need more information. He needed information he could use.

That's where the mapmaker came in: He didn't invent the world. He
formatted it.

The king needed for him to take something vast and make it
\emph{holdable}.

That's what format does in a prompt.

Lest we meet the mapmaker's fate, we must ask ourselves:

``What am I really asking for---and what should it look like when it
arrives?''

When you tell the model what shape you want your map to take, you do
more than make it look nice.

You:

\begin{itemize}
\tightlist
\item
  Make it easier to use
\item
  Facilitate decision-making
\item
  Make it more satisfying to read
\item
  Make it easier to share
\end{itemize}

Because how something is presented matters as much as what's inside: A
list. A summary. A grid. A table. A single sentence.

The right format doesn't just display your request. It translates it
into impact.

Into a map a king or queen could use to defend and develop the kingdom.

\subsection{\texorpdfstring{\textbf{Format a Useful
Map}}{Format a Useful Map}}\label{format-a-useful-map}

If the king had a keyboard instead of a crown, his prompt might have
looked something like this:

``Summarize the territory I control, including trade routes, geography,
rival region borders, and alliances, into a one-page visual map I can
study quickly.''

That's a format-driven prompt.

He isn't asking for more knowledge. He's asking for presentation.
Sometimes what you want is already known--- but it doesn't feel known
because it's not in a usable shape.

The model can produce almost anything. But if you don't tell it what
form you want, you're asking it to pour water without a glass.

Format is the glass. The bowl. The blueprint. The slide deck. The tweet.
The list of ten. The table with three columns and ordered rows.

It's the delivery mechanism for the idea. And when you ask for it well,
you don't just get content. You get clarity.

When we write:\emph{``Tell me what to do.''} The AI might give you:

\begin{itemize}
\tightlist
\item
  A bulleted list
\item
  A philosophical treatise
\item
  A sarcastic haiku
\end{itemize}

If all you want is a checklist. Or a short email. Or a pep talk. But you
don't say the format. The machine---like the player---just \emph{runs it
again}.

Here are some types of format prompts, for when form matters as much as
meaning.

\subsection{\texorpdfstring{\textbf{Format
Types}}{Format Types}}\label{format-types}

Prompts can mold content into many shapes using formats. Here are some
examples:

\subsubsection{\texorpdfstring{\textbf{1. List
Format}}{1. List Format}}\label{list-format}

\textbf{Prompt:}

``List 5 creative ways to open a leadership retreat.''

\textbf{Output:}

\begin{enumerate}
\def\labelenumi{\arabic{enumi}.}
\tightlist
\item
  Start with a blindfolded team walk into the room, then reveal a shared
  challenge.
\item
  Use an anonymous pre-written ``fears and hopes'' box and read a few
  aloud.
\item
  Begin with the most difficult question the group will face---no small
  talk.
\item
  Share one-minute audio messages from team members' colleagues saying
  what they admire about them.
\item
  Start in silence: give everyone one sticky note, one word, and one
  minute.
\end{enumerate}

\subsubsection{\texorpdfstring{\textbf{2. Paragraph
Format}}{2. Paragraph Format}}\label{paragraph-format}

\textbf{Prompt:}

``Write a short paragraph explaining why leaders should embrace
vulnerability.''

\textbf{Output:}

Leaders who show vulnerability invite trust. By acknowledging
uncertainty or failure, they create psychological safety---and signal
that it's okay to be real, to be human. This doesn't weaken
authority---it deepens connection and resilience.

\subsubsection{\texorpdfstring{\textbf{3. Table
Format}}{3. Table Format}}\label{table-format}

\textbf{Prompt:}

``Compare the risks and benefits of in-person vs.~remote employee
onboarding in a table.''

\textbf{Output:}

\begin{longtable}[]{@{}
  >{\raggedright\arraybackslash}p{(\linewidth - 4\tabcolsep) * \real{0.1071}}
  >{\raggedright\arraybackslash}p{(\linewidth - 4\tabcolsep) * \real{0.4762}}
  >{\raggedright\arraybackslash}p{(\linewidth - 4\tabcolsep) * \real{0.4167}}@{}}
\toprule\noalign{}
\begin{minipage}[b]{\linewidth}\raggedright
Format
\end{minipage} & \begin{minipage}[b]{\linewidth}\raggedright
Benefits
\end{minipage} & \begin{minipage}[b]{\linewidth}\raggedright
Risks
\end{minipage} \\
\midrule\noalign{}
\endhead
\bottomrule\noalign{}
\endlastfoot
In-Person & Builds rapport quickly, stronger culture & Travel costs,
limited flexibility \\
Remote & Scalable, more accessible & Harder to build trust,
distractions \\
\end{longtable}

\subsubsection{\texorpdfstring{\textbf{4. Headline / One-Liner
Format}}{4. Headline / One-Liner Format}}\label{headline-one-liner-format}

\textbf{Prompt:}

``Give me 3 headline options for a blog post about AI in education.''

\textbf{Output:}

\begin{enumerate}
\def\labelenumi{\arabic{enumi}.}
\tightlist
\item
  Smarter Classrooms Start With Smarter Questions
\item
  AI Won't Replace Teachers, But It Will Change Them
\item
  Rewriting the Curriculum---One Algorithm at a Time
\end{enumerate}

\subsubsection{\texorpdfstring{\textbf{5. Multi-Part Structured
Format}}{5. Multi-Part Structured Format}}\label{multi-part-structured-format}

\textbf{Prompt:}

``Generate 2 taglines for a mindfulness app. For each, include the
tagline, a rationale, and the intended audience.''

\textbf{Output:}

\begin{enumerate}
\def\labelenumi{\arabic{enumi}.}
\tightlist
\item
  \textbf{Tagline}: Quiet the Noise, Find Your Rhythm
  \textbf{Rationale:} Emphasizes calming internal chatter while
  suggesting personal pace. \textbf{Audience:} Burnt-out professionals,
  ages 30--45.
\item
  \textbf{Tagline:} Stillness Is a Superpower \textbf{Rationale:} Turns
  rest into strength, using superhero metaphor. \textbf{Audience:}
  Ambitious achievers new to mindfulness.
\end{enumerate}

\subsubsection{\texorpdfstring{\textbf{6. Slide / Outline
Format}}{6. Slide / Outline Format}}\label{slide-outline-format}

\textbf{Prompt:}

``Turn this paragraph into a 3-slide outline: `Our company is focused on
sustainable logistics, partnering with urban delivery services to reduce
emissions. We've deployed AI routing, reduced idling times, and
incentivized low-emission fleets.'\,''

\textbf{Output:}

\textbf{Slide 1:} \emph{Our Sustainability Mission}

\begin{itemize}
\tightlist
\item
  Urban delivery partnerships
\item
  Emission reduction focus
\end{itemize}

\textbf{Slide 2:} \emph{Tech That Powers the Shift}

\begin{itemize}
\tightlist
\item
  AI routing optimization
\item
  Idling time reduction
\end{itemize}

\textbf{Slide 3:} \emph{Incentives for Greener Fleets}

\begin{itemize}
\tightlist
\item
  Tiered rewards system
\item
  Long-term impact forecast
\end{itemize}

\subsubsection{\texorpdfstring{\textbf{7. Creative
Format}}{7. Creative Format}}\label{creative-format}

\textbf{Prompt:}

``Explain blockchain as a bedtime story for a curious 7-year-old.''

\textbf{Output:}

Once upon a time, there was a magical notebook that everyone in the
kingdom could see, but no one could erase. Every time someone bought a
toy, shared a snack, or traded marbles, the notebook wrote it down---so
everyone knew it was fair. That notebook was called the
\emph{blockchain}, and it helped the kingdom stay honest, even when the
lights went out.

Each of these format types shapes how the model delivers the answer---
not just in style, but in usefulness. You're not just asking it to talk.

You're asking it to organize, present, and persuade. You're asking for
it to become easier to understand, use, share, or sell.

The format you choose depends upon:

\begin{itemize}
\tightlist
\item
  Who will read it
\item
  How much detail they need
\item
  What they will do with it
\end{itemize}

And the clearer you are about that, the sharper the model's answer will
be.

But just as format fell short in the king's castle, format can fumble in
the meeting room.

\subsection{\texorpdfstring{\textbf{The Invisible
Insight}}{The Invisible Insight}}\label{the-invisible-insight}

It was his first strategy presentation. Nico had done the work---deep
work. He pulled numbers from three systems, tracked customer churn
across six months, talked to sales reps, and distilled it all into one
sharp insight:

``Our onboarding sequence was bleeding users before day three.''

This was real. Important. Fixable. He set out to make his case and ring
the alarm bells. His ChatGPT prompt read: ``Write a summary of our user
onboarding process, drop-off rates, and possible improvements based on
internal interviews and Salesforce data.''

He attached his Salesforce documents and the model delivered--- a tidy,
professional, five-paragraph block of strategy analysis. It was good.

Too good. No bullets. No highlights. No callouts. Just text. Which he
copied it straight into slide 4.

At the meeting he spoke over it---adding more explanation.

The execs squinted. Someone flipped backward a slide. One nodded like a
human screensaver. And then\ldots{} they moved on. No one pushed back.
No one asked for follow-up. No one remembered the insight.

Nico had found the leak. Named it. Outlined the fix. But he never asked
the model: ``Help me shape this in a way they'll actually see.''

No formatting. No contrast. No takeaway. No friction. No fuel.
Just\ldots{} copy.

And the worst part? Nico walked out thinking that maybe it hadn't been
worth presenting at all.

\subsection{\texorpdfstring{\textbf{Takeaways}}{Takeaways}}\label{takeaways-2}

Nico wasn't wrong. He found a meaningful pattern in the data. He
identified a clear cause. He even prompted the model clearly and
concisely.

The request was solid. The insight was real. But insight doesn't land if
no one can \emph{hold} it, like the map.

That's the brutal truth of communication: If you don't shape your
answer, it could become invisible.

Nico forgot that execs don't read paragraphs. They scan. They
prioritize. Context over content. He needed to say: ``Help me format
this so they see it fast.'' Or: ``Turn this into a 3-slide outline with
bold headlines and clear next steps.'' Or even: ``Summarize this insight
in one sentence, one number, and one call to action.''

With a few extra words, Nico could've asked for:

\begin{itemize}
\tightlist
\item
  A key stat in bold at the top
\item
  A 3-part slide structure: Problem, Insight, Fix
\item
  A short list of recommendations
\item
  A one-sentence summary that sticks in the room
\end{itemize}

Even better: ``Write a slide headline that would make someone want to
pause and ask a question.'' That's not formatting for style. That's
engagement. It's not about dumbing it down. It's about framing it up.
Because smart people still need help seeing what matters. And the better
your format, the better your signal gets through the noise.

After Nico's meeting, Elena did the sales strategy presentations. Nobody
said it out loud. But when the next deck needed a clear story, clear
structure, clear slides\ldots{} It was her name they put on the
calendar.

And Nico? Well---he got better at making himself heard. Eventually.

\subsection{\texorpdfstring{\textbf{Reflection}}{Reflection}}\label{reflection-1}

Clarity isn't just about content. It's about what the eye sees first.
What the mind grabs hold of. What the room \emph{remembers.}

We've seen how format can:

\begin{itemize}
\tightlist
\item
  Make ideas \textbf{scannable}
\item
  Turn paragraphs into \textbf{slides}
\item
  Distill complexity into \textbf{a table}
\item
  Translate insight into \textbf{a one-liner}
\item
  Bring a voice to life as \textbf{a fable, a tweet, a script}
\end{itemize}

We've also seen what happens when it's missing--- when good thinking
gets buried because it was presented like a brick instead of a door.

The prompt isn't just about the what. It's about making the what
\emph{usable}.

\subsection{\texorpdfstring{\textbf{Backlit}}{Backlit}}\label{backlit}

You've asked for something. You've shown what it should look like.
You've told it how to shape the answer.

But there's still a question lingering in the model's mind:

``What am I supposed to know before I begin?''

That question leads us to context--- the background, the backstory, the
ingredients behind the request. Because a model can be clear and
clever\ldots{} but if it doesn't have context, it's guessing in the
dark.

Let's turn on the light.

gellitos

\section{Chapter 4: What's the
Context?}\label{chapter-4-whats-the-context}

Context is the invisible lens. It tells the model not just what to say,
but why and for whom. Without context, prompts are blunt. With it, they
become surgical. You don't always need long instructions---just smart
ones that let the model ``see'' what you're doing. For example, here are
some statements:

\textbf{``That's the third one this week.''}

Is that good? Is it bad? Is it too many? Are we talking about
sales\ldots{} or broken chairs?

\textbf{``Okay\ldots{} let's send it.''}

A green light? A gamble? A joke? Did someone finally finish something,
or are they giving up? It could mean \emph{hope it doesn't crash}.

\textbf{``She said yes.''}

To a raise? A job offer? A second date? Is this the start of something
new or years in the making?

\subsubsection{Do those statements make
sense?}\label{do-those-statements-make-sense}

Of course they don't.

Because without context, language becomes a screen with the brightness
turned down.

The words are there. But the meaning hasn't loaded.

Let's turn the lights back on:

\textbf{``That's the third one this week.''} Context: Three clients
signed contracts after the new demo went live. That's three new
accounts, three wins, and the best week the team's had in a quarter.

\textbf{``Okay\ldots{} let's send it.''} Context: After four weeks of
building, rewriting, and debating every headline, the growth team is
finally launching the rebrand. And yeah, they're nervous. But they're
ready.

\textbf{``She said yes.''} Context: After dinner, on the walk home,
under the quiet hum of a streetlight. They'd been together five years.
He asked. She smiled. And said yes.

\textbf{Context is the frame.} It's not the picture, but it helps you
see it clearly.

Without it, prompts fall flat. With it, even short requests can hit the
mark.

\subsection{\texorpdfstring{\textbf{What Is Context in a
Prompt?}}{What Is Context in a Prompt?}}\label{what-is-context-in-a-prompt}

Context is background information that helps the model interpret your
request. It's everything the model needs to know before it starts
answering.

It answers questions like:

\begin{itemize}
\tightlist
\item
  What's this about?
\item
  Who is this for?
\item
  What's already happened?
\item
  What should be included, avoided, or emphasized?
\item
  What matters here?
\end{itemize}

You can think of context as the quiet paragraph before the ask. It says:
``Just so you know\ldots''

\subsection{\texorpdfstring{\textbf{Context Can
Be:}}{Context Can Be:}}\label{context-can-be}

\begin{itemize}
\tightlist
\item
  A business situation
\item
  A snippet of dialogue
\item
  A product description
\item
  A summary of a meeting
\item
  A link to prior performance
\item
  A note about the audience
\item
  A document you uploaded
\item
  A user persona
\item
  A vibe
\end{itemize}

Context can be one sentence. Or it can be an attached strategy deck.

The more relevant and focused the context, the \textbf{smarter} the
response.

\subsection{\texorpdfstring{\textbf{Elena in the Briefing
Room}}{Elena in the Briefing Room}}\label{elena-in-the-briefing-room}

Elena led communications for a fast-growing energy startup. They were
smart. Funded. Scaling faster than their strategy. And this morning, her
CEO had cornered her outside the elevator with a familiar bombshell:

``Can you get me a killer narrative for the investor deck? Something
fresh. We need to punch above our weight.''

No timeline. No brief. Just \emph{punch above our weight}.

Elena nodded, walked to her desk, and opened a blank doc.

Ten minutes later, she opened ChatGPT and typed:

``Write a compelling narrative for a cleantech investor pitch.''

The result was\ldots{} fine. A couple of buzzwords. Something about
disruption. She could almost hear a roomful of distracted VCs checking
their watches.

She deleted it. Took a breath. Then did what she'd been learning to do.

She prompted again:

``Here's the situation: We're a Series A cleantech company with a
working prototype but no revenue. We just got picked up by a national
accelerator. We're asking for \$12M to expand operations. The CEO is
visionary, but we lack traction. We need to signal future dominance
without overselling.''

``Write an investor narrative that frames us as a high-risk,
high-insight bet---`We saw it before others did.' Make it tight and
charismatic.''

This time, the response gave her something she could feel. It told a
story about identifying shifts in grid infrastructure before the big
players caught on. It made the risk sound like insight. It gave her the
spine of a speech she could reshape with the CEO's voice.

Later that day, she used a variant of the same prompt to generate:

\begin{itemize}
\tightlist
\item
  A 3-slide summary
\item
  A 20-second verbal pitch
\item
  And a LinkedIn teaser for the launch announcement
\end{itemize}

What she'd really written wasn't copy. It was clarity. She hadn't just
asked for \emph{words}---she'd asked for \emph{what the words needed to
do}.

And that made all the difference.

\subsubsection{\texorpdfstring{\textbf{Takeaways}}{Takeaways}}\label{takeaways-3}

\begin{itemize}
\tightlist
\item
  A vague ask yields a vague answer---even if the words sound right.
\item
  The more context you bring to your request, the less rework you'll
  face.
\item
  A good request tells the model \textbf{why the words matter}---not
  just what kind you want.
\end{itemize}

A request is not a suggestion. It is an invocation.

The moment you type your prompt, you are not just telling ChatGPT what
to do---\emph{you are shaping how it thinks}.

\subsubsection{\texorpdfstring{\textbf{How to Include Context in a
Prompt}}{How to Include Context in a Prompt}}\label{how-to-include-context-in-a-prompt}

There are three primary ways to include your context into a GPT prompt.
We're not talking about a sophisticated automation or enterprise
repository setup where all kinds of timely and appropriate data could be
automatically and invisibly included in a prompt, often called Retrieval
Augmented Generation(RAG). We're talking about your powerful fingers
typing characters and words directly into the GPT prompt on your
screen.:

\subsubsection{\texorpdfstring{\textbf{1. Inline
Context}}{1. Inline Context}}\label{inline-context}

Put the background directly into the prompt:

``We're launching a budgeting app for college students. It rewards small
wins and celebrates consistency, not just big goals.

Write a landing page headline that captures that tone.''

The model knows the request. But now it knows the product, the values,
and the vibe.

Note that in this book, we \textbf{\emph{never}} assume the model has
context unless we show you how it's clearly provided.

\subsubsection{\texorpdfstring{\textbf{2. Uploaded Context
}}{2. Uploaded Context }}\label{uploaded-context}

If you're using ChatGPT with file support, you can say:

``I've uploaded our Q3 customer feedback report. Based on that, suggest
three new onboarding improvements for the mobile app.''

The context lives outside the prompt, but informs it directly (with
uploaded documents, tables, and PDFs). If you don't upload it or paste
it, it doesn't exist.

\subsubsection{\texorpdfstring{\textbf{3. Running
Context}}{3. Running Context}}\label{running-context}

Sometimes, you can build context over time:

\textbf{You:} ``Here's what we offer\ldots'' \textbf{You:} ``Here's the
audience we're targeting\ldots'' \textbf{You:} ``Okay, now write a short
ad using that info.''

This kind of ``threaded context'' works \emph{in-session}. But the
moment you refresh, sometimes it's gone. So in best practice?
\textbf{Include what matters every time.}

\subsection{\texorpdfstring{\textbf{The Rule of
Context}}{The Rule of Context}}\label{the-rule-of-context}

If you want the model to respond like it knows what you're talking
about, you have to tell it what you're talking about.

Assume nothing. Supply everything that matters. Keep it tight. Keep it
relevant. Keep it close to the request.

Because the model isn't psychic. But it \emph{is} brilliant---when it
knows where to start.

\subsection{\texorpdfstring{\textbf{The Letter She Couldn't
Start}}{The Letter She Couldn't Start}}\label{the-letter-she-couldnt-start}

She had been staring at the blinking cursor for hours. The screen was
blank. The memories weren't.

Elena hadn't spoken to her sister in over two years---not since the
funeral, not since the argument about the house. Now, she wanted to say
something. Not everything. Not an apology, not exactly. Just something
that could begin a thaw.

She opened ChatGPT and typed:

``Write a message to my sister.''

The model responded instantly:

\emph{Hi there! Hope you're doing well. Just wanted to check in and see
how things are going!}

Elena stared at the chipper nothing on her screen. She deleted it.

She tried again:

``Write a message to reconnect with my sister after a long silence.''

This time it came back soft. Too soft.

\emph{It's been a long time. I hope you're doing okay. I'd love to catch
up sometime.}

It wasn't wrong. But it didn't sound like her. It didn't carry the ache,
or the hesitation, or the small shard of pride she hadn't yet dislodged.

She sat back, frustrated. And then---without fully realizing it---she
began to \emph{prompt differently}.

\textbf{``Here's the situation: I haven't spoken to my sister in two
years after a family falling out. I want to send a message that feels
honest but gentle, with just enough warmth to invite her in. I don't
want to get into everything---we're not ready. I just want to open the
door.''}

\textbf{``Write the message in my voice. I tend to be concise, a little
dry, but sincere.''}

The response came back slowly. Measured. Cautious. Familiar.

\emph{Hey. I've been thinking about you. I know there's a lot we haven't
said---and maybe still shouldn't---but I'd really like to reconnect,
even just a little. No pressure. Just letting you know the door's open.}

Elena read it once. Then again. Then added one line of her own:

\emph{Only if you want to.}

She hit send.

And for the first time in a long while, it felt like a conversation had
begun.

\subsection{\texorpdfstring{\textbf{Why the Third Prompt
Worked}}{Why the Third Prompt Worked}}\label{why-the-third-prompt-worked}

Elena's third prompt succeeded because it named the relationship. It
gave emotionally anchored details and defined the desired tone. It
included a sample of the sister's own voice---her actual words. And it
described the occasion (her birthday).

Elena's sharing gave the model context: ``Here's the situation\ldots''

The model responded with the words she needed to move forward with her
sister. Move towards the start of something true.

\subsection{\texorpdfstring{\textbf{Chapter Reflection:
Context}}{Chapter Reflection: Context}}\label{chapter-reflection-context}

You've learned how to ask. You've learned how to show. You've learned
how to shape.

But none of it works if the model doesn't know what's behind the prompt.

Context is what brings a prompt into the real world. It turns ideas into
understanding. It fills in what the model can't guess.

Who's it for? What's at stake? What's the moment? What matters?

Context answers these questions so the model can respond like it knows
where it is.

We've seen it at work:

\begin{itemize}
\tightlist
\item
  In launch campaigns
\item
  In a sensitive email
\item
  In requesting guidance
\item
  In pitches, slide decks, and summaries
\end{itemize}

Context doesn't need to be long but it does need to be present.

It's the map before the mission.

\subsection{\texorpdfstring{\textbf{And Now the Model Asks: ``Who Am
I?''}}{And Now the Model Asks: ``Who Am I?''}}\label{and-now-the-model-asks-who-am-i}

You've told it what to do. You've shown what it should look like. You've
shaped the form. You've shared the world. But there's one more question
the model is ready to ask:

``Who should I be?''

That's where we go next. The chapter on role --- where voice, tone, and
perspective come to life. Because sometimes, it's not what the model
says. It's who's speaking.

gellitos

\section{Chapter 5: Your Assistant's
Role}\label{chapter-5-your-assistants-role}

Telling the model what it is, who it is, or how it should behave unlocks
a different class of interaction. This isn't just tone---it's
perspective. By assigning a role, you invite the model into character.

You can prompt it as a chef, a coach, a philosopher, or a project
manager. Each identity reshapes the answers it gives---even when the
topic stays the same.

``Pretend you're my therapist.'' ``Act like a skeptical investor.''
``Speak as if you were Steve Jobs.'' ``Write this like a friend who's
disappointed but still hopeful.''

These aren't gimmicks. They're invitations.

Invitations for the model to take on a role--- to shift its posture, its
tone, its judgment, even its values.

And when used well, they can turn a flat answer into something alive.

\subsection{\texorpdfstring{\textbf{The Model Doesn't Have a
Voice---Until You Give It
One}}{The Model Doesn't Have a Voice---Until You Give It One}}\label{the-model-doesnt-have-a-voiceuntil-you-give-it-one}

GPT doesn't speak like a person. It speaks like a million people. Which
means its voice is neutral, until you anchor it.

That's what role does. It's how you ask it to:

\begin{itemize}
\tightlist
\item
  Speak with authority
\item
  Offer care
\item
  Use skepticism
\item
  Stay playful
\item
  Get formal
\item
  Get human
\end{itemize}

Not just what to say. How to say it. Who to say it as.

\begin{center}\rule{0.5\linewidth}{0.5pt}\end{center}

\subsection{\texorpdfstring{\textbf{Why Role
Matters}}{Why Role Matters}}\label{why-role-matters}

Every real-world conversation happens in context and in character. You
talk differently when you're:

\begin{itemize}
\tightlist
\item
  A manager giving feedback
\item
  A best friend calming nerves
\item
  A product lead in a roadmap meeting
\item
  A brand, a voice, a mood
\end{itemize}

Without role, the model may sound too stiff. Or too casual. Too salesy.
Or not persuasive enough. Too confident. Or not confident \emph{enough.}

But when you give it the right role, you get responses that don't just
make sense--- they feel right.

\subsection{\texorpdfstring{\textbf{The Pirate, the Butler, and the
CEO}}{The Pirate, the Butler, and the CEO}}\label{the-pirate-the-butler-and-the-ceo}

They were supposed to be doing homework.

But the math sheet lay abandoned on the floor, and ChatGPT was open on
the laptop. Ben, age 10, sat at the keyboard. Caleb, age 8, was draped
across the bed, holding a peanut butter cracker like a microphone.

``Make it talk like a pirate,'' Caleb said. Ben laughed like he'd been
waiting all day for that request.

He typed:``Tell us how to make a sandwich, but do it like a pirate.''
ChatGPT answered:

``Arrr, first ye be takin' two slabs o' bread---none o' that soggy
bilge, mind ye---and layin' 'em down like the planks they are!''

Caleb shrieked. Ben kept typing:

``Now do it like a robot.'' ``Now like a poet.'' ``Now like my big
sister when she's mad.''

Each voice brought new words. New rhythms. New choices. Same sandwich.

At some point, they forgot to laugh. They were reading silently.
Listening. Later that night, when their mom came in, she asked if they'd
finished their math.

``Almost,'' Ben said, ``but\ldots{} did you know sandwiches can sound
like anything?''

\subsection{\texorpdfstring{\textbf{What the Boys Were Really
Learning}}{What the Boys Were Really Learning}}\label{what-the-boys-were-really-learning}

Ben and Caleb weren't playing with lunch instructions. They were playing
with voice.

They were discovering --- through laughter and language --- that
\emph{how} something is said changes what it feels like. Even when the
facts stay the same.

Each time they changed the role --- they shifted the tone, the pacing,
the intention.

The pirate gave commands. The robot was sterile. The poet felt it. His
big sister meant it.

Same request. Different voice. Different effect. They were learning the
lesson professionals often forget:

Voice shapes meaning. Role shapes response.

Now, let's turn that play into power. Now that we've seen what two kids
can teach us about role, let's step into the real world again, where
voice isn't just entertainment --- it's strategy.

Here's a spread of practical, real-world role prompts, each one showing
how the same request changes when the model is asked to speak
differently:

\subsection{\texorpdfstring{\textbf{1. The
Coach}}{1. The Coach}}\label{the-coach}

\textbf{Prompt:} ``Give me advice on how to handle a rough week at work.
Speak like a coach who believes in me but won't let me off the hook.''

\textbf{Response:} You had a rough week. Fine. You're allowed one. But
now we get up. What did you learn? What will you do differently Monday
morning? Take thirty minutes, make a plan, and then get out of your own
way.

\subsection{\texorpdfstring{\textbf{2. The
CEO}}{2. The CEO}}\label{the-ceo}

\textbf{Prompt:} ``Explain this new product to potential investors.
Speak like a confident, visionary CEO.''

\textbf{Response:} We're not just building a tool---we're redefining how
companies think about knowledge flow. What we've created is a way to
turn scattered expertise into real-time intelligence. And the market is
begging for it.

\subsection{\texorpdfstring{\textbf{3. The
Skeptic}}{3. The Skeptic}}\label{the-skeptic}

\textbf{Prompt:} ``Evaluate our new marketing tagline. Act like a
skeptical customer who's seen it all.''

\textbf{Response:} ``Smarter solutions for a better tomorrow''?
Seriously? Sounds like ten other SaaS companies I ignored this week.
What makes yours different---and why should I trust you?

\subsection{\texorpdfstring{\textbf{4. The
Therapist}}{4. The Therapist}}\label{the-therapist}

\textbf{Prompt:} ``Help me reframe my anxiety about switching jobs. Use
the voice of a compassionate but honest therapist.''

\textbf{Response:} It's completely normal to feel fear in the face of
change. But fear doesn't mean you're wrong---it means something
important is happening. Let's unpack where the fear's coming from, and
what it might be trying to protect.

Each of these prompts asks for roughly the same thing: help, advice,
explanation, feedback.

But the voice changes the \emph{relationship}. The model stops being a
neutral observer. It becomes a coach, a confidant, a leader, a skeptic,
a teacher, a narrator.

That's the magic of role. That's why role matters.

\subsection{\texorpdfstring{\textbf{How to Assign Role in a
Prompt}}{How to Assign Role in a Prompt}}\label{how-to-assign-role-in-a-prompt}

Assigning a role means explicitly telling the model who it should
pretend to be while answering.

The model will answer any question you give it. But unless you tell it
who to be, it won't know how to speak. Role is the part you cast the
model in. It's the voice it wears, the posture it takes, the kind of
mind it pretends to have while it speaks.

It's easy to assign. You simply say:

``Speak like a mentor.'' ``Answer like a friend.'' ``Respond as a
therapist, kind but direct.'' ``Explain this like a tired professor who
still cares.''

You can place that request anywhere in the prompt. At the beginning, in
the middle, or right at the end. What matters is that you say
it---clearly. Because the model won't remember your tone for long. It
might carry over a turn or two, but the moment you shift topics, it
forgets who it was pretending to be.

If voice matters---assign it each time! And when role is missing, the
response can land soft and gray. Helpful, sure. But generic.
Forgettable. No rhythm. No edge. No soul. Ask something heartfelt, and
you'll get something formal. Ask something urgent, and you'll get a
list. Ask for help, and it will hesitate, not because it doesn't care---
but because you didn't show it how to speak.

Role doesn't just change the model's voice. It changes its judgment, its
values, its tone and its timing. It changes what gets left out. What
gets softened. What gets spotlighted.

Without role, you get output.

With role, you get presence.

\subsubsection{\texorpdfstring{\textbf{Use Any Role You Can Think
Of}}{Use Any Role You Can Think Of}}\label{use-any-role-you-can-think-of}

``Act as a \hyperref[role]{role}.'' ``Speak like a {[}persona{]}.''
``Respond in the voice of a {[}type of person{]}.'' ``Imagine you are a
{[}professional / identity / archetype{]}.'' ``Write this as if you are
{[}personality or relationship role{]}.''

Here are some examples:

``Act like a startup mentor giving tough love.'' ``Speak like a Gen Z
TikToker hyping a product.'' ``Respond like a senior policy advisor
under pressure.'' ``Write as if you're a parent talking to a worried
child.'' ``Explain it like a stand-up comic who's also good at math.''

These can go at the beginning or end of a prompt. Both work.

\subsection{\texorpdfstring{\textbf{Do Role and Request Go in the Same
Prompt?}}{Do Role and Request Go in the Same Prompt?}}\label{do-role-and-request-go-in-the-same-prompt}

Yes. Best practice is to assign Role within the same prompt as your
Request.

Why? Because:

\begin{itemize}
\tightlist
\item
  It provides context in the moment
\item
  It eliminates reliance on memory
\item
  It ensures the model is aligned with that specific response
\end{itemize}

You can assign role at the top:

``Act like a kind but firm manager. Now write an email giving feedback
on missed deadlines.''

Or at the bottom:

``Give me 3 subject line options for this product launch. Respond like a
creative director at an ad agency.''

\subsection{\texorpdfstring{\textbf{Will ChatGPT Remember the
Role?}}{Will ChatGPT Remember the Role?}}\label{will-chatgpt-remember-the-role}

Sometimes, but not reliably, although GPT memory improves with each
model release.

If you're in a single thread and the tone continues, it may loosely
carry over. But role is fragile. It degrades over turns, especially if
you change topics or reset tone.

If voice matters, restate the role. Don't assume it'll remember. It
won't persistently simulate a persona unless you remind it.

\begin{center}\rule{0.5\linewidth}{0.5pt}\end{center}

\subsection{\texorpdfstring{\textbf{What Happens If You Skip the
Role?}}{What Happens If You Skip the Role?}}\label{what-happens-if-you-skip-the-role}

If you don't assign a Role, the model will respond with:

\begin{itemize}
\tightlist
\item
  Neutral tone
\item
  Formal structure
\item
  Slightly generic phrasing
\item
  Mild assumptions based on common patterns
\end{itemize}

It's not wrong. But it's often bland, beige, and forgettable. Think
default Wikipedia voice. Useful, but without eloquence.

In creative work, this makes it fall flat. In business, it can sound
vague or misaligned. In emotional contexts, it might come off cold or
robotic.

\subsection{\texorpdfstring{\textbf{How Can Role Be
Misused?}}{How Can Role Be Misused?}}\label{how-can-role-be-misused}

Roles can be used poorly. Here how:

\begin{enumerate}
\def\labelenumi{\arabic{enumi}.}
\item
  \textbf{Vague Roles}

  ``Act like a good writer.'' is open-ended. What kind of good writer?
  Poetic? Technical? Journalistic?
\item
  \textbf{Conflicting Roles and Requests}

  ``Act like a strict lawyer. Now write a love letter.'' Unless
  intentional for irony, this creates tone confusion.
\item
  \textbf{Overly Prescriptive Roles}

  ``Act like my therapist, but use no emotion, contractions, or
  metaphor.'' Now you've handcuffed the model into an unnatural voice.
\item
  \textbf{Inconsistent Role Use}

  ``Sell this product. Act like a relaxed, helpful salesman'', then
  ``Push the product aggressively.'' Switching tone drastically without
  reassigning role or resetting voice.
\end{enumerate}

\subsection{\texorpdfstring{\textbf{The Letter That Didn't
Land}}{The Letter That Didn't Land}}\label{the-letter-that-didnt-land}

Priya had just been promoted.

It had been a quiet climb---years of steady work, no drama, no
spotlight. But now, after five years, she was stepping into a team lead
role. Her first task: send a note to the team she'd just been asked to
lead.

She opened ChatGPT and typed:

``Write an announcement email that says I've been promoted to team lead.
Be polite, brief, and professional.''

It delivered exactly what she asked for.

\textbf{Subject: Leadership Transition}

``I'm pleased to share that I'll be stepping into the role of Team Lead
effective Monday. I'm grateful for the opportunity and look forward to
continuing our work together. Please feel free to reach out as we
navigate this transition.''

It was correct. Clean. Balanced. She read it twice. It felt\ldots{}
safe. She sent it.

No replies. No congratulations. No comments at all.

Just silence.

That afternoon, she overheard someone say, ``I didn't even know it was
her. Sounded like something HR wrote.''

The truth? She hadn't asked the model to write like \emph{her}. Or like
a peer. Or even like a human with a voice. She'd asked for ``brief and
professional.'' And that's what she got. The model didn't fail. The
prompt did. Not because of the words.

But because of the absence of a role.

\subsection{\texorpdfstring{\textbf{Post-Game: The Voice That Wasn't
Invited}}{Post-Game: The Voice That Wasn't Invited}}\label{post-game-the-voice-that-wasnt-invited}

Priya didn't do anything wrong.

She was polite. Professional. Appropriate. But those aren't the
qualities that make people lean in. They're the ones that make people
scroll past. Her note had no fingerprints on it. No trace of who she
was. No rhythm, no warmth, no wink of familiarity.

She wrote like someone trying not to make a mistake. Which meant no one
felt invited to celebrate with her.

And all it would've taken was this:

``Write an announcement email letting my team know I've been promoted to
team lead. Keep it warm and humble. Speak like a colleague who's
genuinely grateful---and maybe a little nervous. Add a sentence that
shows I'm excited to support them, not just lead them.''

That's not a fancy prompt. It's a clear one. It defines the who---not
just the what.

And here's what the model might've returned:

\textbf{Subject: I Get to Brag About You Now}

``Just a quick note to share that I've been asked to step into the team
lead role starting Monday. I've been learning from all of you for the
last five years, and I still am---so stepping into this new role feels a
little surreal (and a little terrifying, let's be honest).

I'm genuinely excited to support this team---not just from a new seat,
but as someone who still remembers what it felt like to join it. I want
to keep what's working, and keep listening to what's not.

Thanks for the kind words (in advance or eventually). My door's open.
Always has been.''

That's voice. That's presence. Same news. Different impact.

Because the role she played in that prompt wasn't ``new manager.'' It
was human being with something to say.

And that's who people respond to.

\subsection{\texorpdfstring{\textbf{The
Takeaway}}{The Takeaway}}\label{the-takeaway}

Role gives the model permission to adopt a voice. Without it, you're
talking to a chorus with no soloist.

Use role when:

\begin{itemize}
\tightlist
\item
  Tone matters
\item
  Authority matters
\item
  Personality matters
\item
  Relationship dynamics matter
\end{itemize}

And if in doubt: Just say who you want the model to be. It'll do the
rest---beautifully.

\subsection{\texorpdfstring{\textbf{Chapter Reflection:
Role}}{Chapter Reflection: Role}}\label{chapter-reflection-role}

The model can mimic any voice --- but only if you ask for one.

We've seen what happens when you give it the wrong tone, or no tone at
all. Even good information can fall flat if the voice behind it feels
distant, generic, or worse---inauthentic.

But with the right role --- a mentor, a friend, a critic, a poet--- the
model speaks like it knows your audience. Because it does. You showed it
how.

Role isn't decoration. It's how you shape authority, intimacy, tone,
posture, rhythm, restraint.

Sometimes it's bold. Sometimes it's subtle. But it always matters.

Because a request with no voice is a story with no narrator.

\subsection{\texorpdfstring{\textbf{Are There Any
Limits?}}{Are There Any Limits?}}\label{are-there-any-limits}

You've told it what you want. You've shown it how it should sound.
You've given examples. You've provided background. You've chosen the
voice. But what about the edges? The boundaries? What's allowed? What's
off-limits?

That's where we go next: constraints. The art of saying: ``Yes---but
only like this.''

gellitos

\section{Chapter 6: What are the
Constraints?}\label{chapter-6-what-are-the-constraints}

Constraint is not limitation---it's leverage. It's the sharp edge that
gives shape to infinite possibility. Want fewer words? Set a word count.
Want only rhyming answers? Say so. Constraints make outputs smarter,
tighter, and often far more creative.

This chapter shows how to rein in ChatGPT not to limit its ability, but
to focus it:

``Summarize this article in three bullet points, no more than 12 words
each. No adjectives.'' ChatGPT might respond:

\begin{itemize}
\tightlist
\item
  \emph{Users abandoned app after price increase}
\item
  \emph{Competitor launched similar product last week}
\item
  \emph{Support team overwhelmed by refund requests}
\end{itemize}

That's a boardroom slide, not a blog post. Clean, cold, direct---because
you asked it to be:

``Rewrite this paragraph for a sixth-grade reading level. Keep it
friendly.'',

for a response like this:

Sometimes, our brains get really busy. When that happens, it's hard to
focus. Taking deep breaths or a short walk can help you feel better.

Not dumber. Just clearer. Constraint ≠ simplification. It's refinement:

``Write a welcome email in under 60 words. Use no exclamation marks.
Make it warm without sounding salesy.'' might get you:

Welcome aboard. We're glad to have you with us. If you ever need
anything, just reach out. We're here to help---quietly and completely.

That's not just text. That's tone control:

``Generate five startup ideas in ten words or less. Avoid the words `AI'
or `blockchain.'\,'' could yield:

\begin{itemize}
\tightlist
\item
  Subscription snacks based on your DNA
\item
  Digital wardrobe that plans outfits
\item
  Kids' bank that teaches saving through stories
\item
  App that rates your posture in real time
\item
  Tool for recording family history, one voice message at a time
\end{itemize}

Constraints don't limit creativity. They guide it. Like banks on a
river---they give the current its power.

\subsection{\texorpdfstring{\textbf{Reveal What
Matters}}{Reveal What Matters}}\label{reveal-what-matters}

You don't constrain the model to make it smaller. You constrain it to
make it more precise. More useful. More like what you meant in the first
place.

Because the real enemy of clarity isn't error---it's excess. Too many
words. Too many ideas. Too many directions at once.

And the model---left alone---will happily give you all of it. Everything
it can imagine, whether or not you need it. But when you add a
constraint --- a limit on length, tone, language, format, scope ---
you're not muzzling the response.

You're shaping it. You're saying: ``No, not everything. Just this.'' And
inside that boundary, something better appears: Something clear.
Something tight. Something \emph{intentional.}

This is where power lives. Not in asking for more. But in asking for
only what matters.

\subsection{\texorpdfstring{\textbf{The Email That Spooked the
Room}}{The Email That Spooked the Room}}\label{the-email-that-spooked-the-room}

Jules ran comms at a venture-backed startup. Cool product. Strong
quarter. They were prepping for a Series B. And now the founder wanted
her to send a little ``touch base'' email to their biggest investor.

Something casual. Light. Just to stay warm.

Jules was tired. She had five other things going. She opened ChatGPT and
typed:

``Write a brief update email to our main investor about recent progress.
Sound confident and friendly.''

She skimmed what came out. Looked fine. Sounded smart. She copy-pasted
it, tweaked a word or two, and hit send.

Four minutes later, the founder was in her Slack:

``Did you tell them we're `experimenting with alternate monetization
frameworks'?''

Jules blinked.

``Did you mean to use the phrase `stealth pivots'??''

The email hadn't been ``fine.'' It had been way too much.

ChatGPT had filled in the gaps. It wanted to be helpful. It took
``progress'' to mean ``strategic evolution,'' and ``confident'' to mean
``positioned for disruption.''

The investor --- sharp, old-school, allergic to nonsense --- had replied
with a single word: ``Elaborate?''

And just like that, a simple touchpoint turned into a two-hour damage
control call because no one told the model what \emph{not} to say.

\subsection{\texorpdfstring{\textbf{Post-Game: What Jules Should've
Done}}{Post-Game: What Jules Should've Done}}\label{post-game-what-jules-shouldve-done}

The prompt she gave was loose:

``Write a brief update email to our main investor about recent progress.
Sound confident and friendly.''

But ``brief'' is subjective. ``Progress'' is vague. ``Confident'' can
spill into arrogant, evasive, or even risky. And with no constraints,
the model tried to fill the silence with swagger. Here's what Jules
\emph{could} have written:

``Draft a short email---no more than 80 words---updating our lead
investor on user growth and the new onboarding feature. Keep it specific
and clear. Avoid buzzwords, changes of course, and finances.''

Now the model knows:

\begin{itemize}
\tightlist
\item
  What to focus on
\item
  What to ignore
\item
  How long to speak
\item
  How not to sound
\end{itemize}

It doesn't have to guess what ``professional'' means. It doesn't reach
for filler. It doesn't invent drama.

It just writes what matters. Here's what that version might've looked
like:

Hi Chris---just a quick update: The onboarding flow is live and already
showing a 12\% lift in activation. MAUs are up 8\% month-over-month. No
major surprises, just steady traction. Hope all's well on your end.

That's it. No hype. No pivot. No stealth. Just facts in clean shoes. And
that would've been the end of it.

No Slack ping. No damage control. No explaining ``alternate monetization
frameworks'' at 6:30 p.m. on a Thursday.

\subsection{\texorpdfstring{\textbf{Constraint
Types}}{Constraint Types}}\label{constraint-types}

A quiet tightening. A necessary narrowing. Not less---just clearer.

The model, left unchecked, will always try to give you everything. It
will pile on options, offer extra lines, give you metaphors, summaries,
analogies, hashtags, disclaimers, sometimes even quotes from people you
didn't ask for. Not because it's wrong, but because it doesn't know
where the edges are.

Constraints are how you draw those edges.

Not to shrink the response, but to give it form. Not to limit what the
model says, but to control how much, how strongly, and in what
direction. Here are the kinds of limits that matter most:

\subsubsection{\texorpdfstring{\textbf{1.
Length}}{1. Length}}\label{length}

Some truths only need a sentence. Others, a single word. If you don't
say how much, the model might say too much.

When you set a length --- by words, lines, bullets, paragraphs --- you
force the model to choose. To cut. To condense. To commit.

It's not about brevity for its own sake. It's about pressure, the kind
that makes language shine:

``Give me three bullets only.'', ``Summarize this in a single
sentence.'', ``No more than 100 words. Keep it tight.'', ``One
paragraph, not two.''

The shorter the shape, the sharper the thought.

\subsubsection{\texorpdfstring{\textbf{2. Tone}}{2. Tone}}\label{tone}

The model doesn't have a voice until you give it one. You don't just
want the right information. You want the right feeling in the room. Tone
constraints don't control what's said. They control how it's heard.

Kind. Casual. Wry. Reverent. Neutral but not cold. Excited but not
obnoxious.

These aren't decorations. They're anchors:

``Make it friendly, but not unprofessional.'', ``Use a gentle
tone---think therapist, not teacher.'', ``Make it sound like I'm
explaining this to a friend, but not angrily.'', ``Respectful. Not
apologetic. Firm.''

Tone can make the difference between read and received.

\subsubsection{\texorpdfstring{\textbf{3. Vocabulary
}}{3. Vocabulary }}\label{vocabulary}

You can say something ten different ways, but the words you choose
decide who understands it.

Vocabulary constraints shape the accessibility of your response. They
tell the model how complex to be, how dense, how plain.

Not all clarity is simple, but simplicity often helps. You can ask for
common words. Or for domain-specific ones. Or for the kind of language a
12-year-old could love. Here are some ideas:

``Use only the most common words.'', ``Avoid jargon---speak plainly.'',
``Use precise legal terms where appropriate.'', ``Explain this as if I'm
new to the field and avoid involved terminology.''

It's not about dumbing down. It's about letting people in.

\subsubsection{\texorpdfstring{\textbf{4.
Scope}}{4. Scope}}\label{scope}

The model knows a little about everything. But that doesn't mean you
want all of it.

Scope constraints tell it where to look --- and where not to wander.
They don't change the answer. They narrow the lens.

You might limit it to a region. Or a moment in time. Or a point of view.

You're not closing doors. You're saying: Just this room. Not the whole
house.

``Focus on the financial side of this.'', ``Keep your answer within the
20th century.'', ``Use examples only from biology.'', ``Answer this like
its a debate in a classroom.''

Scope is how you stay on track and keep the model from taking scenic
detours.

\subsubsection{\texorpdfstring{\textbf{5.
Exclusion}}{5. Exclusion}}\label{exclusion}

Sometimes what matters most is what doesn't belong. Exclusion
constraints tell the model what to leave out. Not just topics. But
phrases. Clichés. Brands. Spoilers. Warnings you've already covered.

They're small but powerful --- the guardrails that prevent a wrong turn
in an otherwise right direction.

``Do not mention any competitors.'', ``Avoid buzzwords like
`cutting-edge.'\,'', ``Leave out your usual safety disclaimers.'', ``Do
not summarize the plot.''

This is how you keep the voice clean. The content focused. The message
intact.

These constraints --- length, tone, vocabulary, scope, exclusion --- are
not limitations. They're \emph{invitations to clarity.}

Each one says: Now that I've asked you what I want, here's how I want
it. Not everything. Just this. And the model --- grateful for the
guidance --- will finally stop guessing and start delivering.

Speaking of guidance, it's time for a parenting allegory.

\subsection{\texorpdfstring{\textbf{The Mother and the Messy
Room}}{The Mother and the Messy Room}}\label{the-mother-and-the-messy-room}

The mother stood in the doorway of her daughter's room. She didn't cross
the threshold. She didn't raise her voice. She just said:

``I need you to clean this up.''

The daughter, who was twelve and dreaming about the collapse of
galaxies, nodded vaguely and returned to sketching something furious in
her notebook. Two hours passed. The room looked the same. Maybe worse.

The mother sighed, came back, said it again:

``I asked you to clean.''

The daughter blinked. ``I did. I put the clothes on the chair.'' The
mother looked at the chair. It had collapsed under the weight of a
half-laundry-mountain. The floor was a terrain of paper, socks, snack
fossils.

The mother was tired. The daughter was confused. They were both angry
and neither understood why.

What the mother didn't say:

``Here's what I expect you to clean and not to clean.'' ``Don't just put
a couple of things away and think the job is done.'' ``Clothes do not
belong on the chair. Put them in the basket or in your bureau.''

The mother made a request but gave no constraints. Models, like
children, are apt to daydream without some discipline.

We say to the model: Now that I've asked you what I want, here's how I
want it. Not everything. Just this. And the model will stop guessing and
start delivering.

Constraints are the final element in prompt anatomy! Let's sum up this
little journey.

gellitos

\section{Chapter 7: How to Prompt
Powerfully}\label{chapter-7-how-to-prompt-powerfully}

There's a moment after the map is drawn, after the training is done,
after the roles are chosen, and the rules are laid down.

A quiet moment.

And in that moment, someone opens a blank chat window, types a single
line, and waits. This chapter is for that moment.

You already know the parts:

\begin{itemize}
\tightlist
\item
  The Request: what you want
\item
  The Example: the pattern to follow
\item
  The Context: where you're standing when you ask
\item
  The Format: the shape you need the answer to take
\item
  The Role: who's speaking and to whom
\item
  The Constraint: what should be held back
\end{itemize}

Each chapter has taught you one. Each story showed how it works. But now
it's time to see how they work together.

Elena didn't know she was teaching you request when she reached out to
ask for help with a product launch, a poem, a life. She asked, again and
again, until the shape of the ask sharpened into a tool.

The King didn't know his mapmaker was teaching format. He just wanted
direction. But in showing him form, she gave him more than
direction---she gave him vision.

The coach taught by doing. She didn't explain what an example was. She
just said: ``Like this.'', ``Try it that way.'', ``Let me show you what
I mean.''

And the little boys? They gave the model voice after voice until it
laughed back at them --- a pirate, a poet, a robot, a sister. They
taught you role without knowing they were doing it.

Jules taught you constraints --- the hard way. When no one said what not
to say, the model filled the silence with smoke.

Alan, Jake, Priya, Nico\ldots{} All of them were speaking with power.
Some knew it. Some didn't. But you know it now.

Because this is the truth of it:

A prompt is not a question. It is a composition.

Like a score for an invisible orchestra. Like stage directions for a
voice with no body. Like a letter sent to the future.

When you prompt powerfully, you are not ``using AI.'' You are crafting
language that shapes response.

You are not waiting to be impressed. You are directing the impression.

There is no perfect prompt. There is only the one that fits the moment.
Tight where it needs to be. Open where it dares to be.

Some prompts are short. Some are careful. Some are playful, strange,
poetic, brave. But the best ones --- the powerful ones --- are
\emph{composed.}

They don't just include a request. They show what the request means.
They don't just assign a role. They know who that role is speaking to.
They don't just limit. They leave room.

Every element influences the others.

A format can suggest a tone. A constraint can force clarity. An example
can replace a paragraph of explanation. A role can collapse the need for
length. A context can light your set.

These aren't tools in a box. They're strings on the same instrument.

And you--- you've been learning to play.

\subsection{The Compleat Prompt}\label{the-compleat-prompt}

Let's see it all, gently, in one ask.

``Act as a seasoned career coach. Based on the attached resume, write a
short email this candidate Evan Pratt can send to a recruiter Gayle
Marquez at a creative agency. Keep it under 100 words. Make the tone
curious and confident. Don't include a subject line. Here's a sample of
how Evan writes.''

There it is:

\begin{itemize}
\tightlist
\item
  \textbf{Role}: a seasoned coach
\item
  \textbf{Context}: a resume is attached
\item
  \textbf{Request}: write an email for a job
\item
  \textbf{Tone}: curious and confident
\item
  \textbf{Constraint}: under 100 words, no subject line
\item
  \textbf{Example}: a sample writing style is provided
\end{itemize}

All in one breath. Nothing wasted. Everything directed. The result?

Will feel like it was written by someone who knows exactly what they
want --- and how to ask for it:

Hi Ms.~Marquez, I came across your agency's work on the Vega
campaign---beautifully done. I'm exploring new creative opportunities
where voice and brand intersect and your team's approach really stood
out. Would love to connect and hear what you're building next.

Warmly, Evan Pratt

Just under 70 words. No filler. Curious, but not overeager. Confident,
but not performative. No buzzwords. No backstory. Just enough for the
door to open. Because the prompt didn't just say what it wanted.

It made the path \emph{walkable.}

\subsection{The Final Prompt}\label{the-final-prompt}

``Act as a poetic and practical author.

The reader is finishing a book about how to write powerful prompts for
large language models.

Based upon the book's teachings, write a final passage that will feel to
the reader like they are closing a guide, not to using a tool, but to
one of the most important skills in life:

Making an effective request.

Use the same voice as the book---simple, lyrical, confident. Limit to
120 words. Do not summarize the book. Do not repeat the chapter titles.
Speak to the reader as someone who now understands the structure of a
prompt as a way of structuring their communication, their instructions,
and their questions.

Leave them with something that feels quiet and true.''

And the model --- having heard everything --- replies:

You know how to ask now. Not just questions, but \emph{the kind that
opens doors.}

You know what to include. What to leave out. How to make the words fit
the moment.

So go ask.

Not always for answers. But for understanding. For clarity. For
connection.

Let the silence before the prompt become part of the message. Let the
way you ask reflect who you are.

And remember--- every great answer was once a better question.

This book ends. The conversation begins.

gellitos

\section{Appendix: 500 Power Prompts}\label{appendix-500-power-prompts}

Here are 500 power prompts arranged in our categories of request,
examples, format, context, role, and constraints.

\subsection{\texorpdfstring{\textbf{Request}}{Request}}\label{request}

Here are 100 prompt request examples in 10 themed subcategories.

\subsubsection{1. Writing \& Editing}\label{writing-editing}

Rewrite, rephrase, and polish like a pro editor.

\textbf{Rewrite for Tone}\\
Make this more assertive.

\textbf{Clarity Check}\\
What parts of this paragraph are unclear?

\textbf{Style Mimicry}\\
Rewrite this in the voice of Hemingway.

\textbf{Jargon Translation}\\
Make this technical doc readable for a 6th grader.

\textbf{Elevator Pitch}\\
Boil this down to a 30-second pitch.

\textbf{Summarize Cleanly}\\
Summarize this in five bullet points.

\textbf{Concise Rewrite}\\
Make this more concise without losing meaning.

\textbf{Simplify Language}\\
Turn this into plain English.

\textbf{Tone Shift}\\
Change this to sound more empathetic.

\textbf{Add Humor}\\
Rewrite this to be lightly humorous without being silly.

gellitos

\subsection{2. Content Creation}\label{content-creation}

Turn rough ideas into compelling content across multiple formats.

\textbf{Blog to Thread}\\
Turn this blog post into a tweet.

\textbf{Headline Variants}\\
Give me 10 headline versions of this idea.

\textbf{Social Adaptation}\\
Rewrite this for a LinkedIn post.

\textbf{Script Starter}\\
Turn this article into a short video script.

\textbf{Lead Magnet}\\
Create a freebie idea based on this content.

\textbf{Newsletter Opener}\\
Write an intro paragraph for this newsletter.

\textbf{Call to Action}\\
Write a compelling CTA for this message.

\textbf{SEO Juice}\\
Add relevant keywords to this article.

\textbf{Punchy Rewrite}\\
Make this more attention-grabbing.

\textbf{Visual Prompt}\\
Describe a featured image that matches this post.

gellitos

\subsection{3. Communication}\label{communication}

Sharpen your message whether you're pitching, persuading, or politely
declining.

\textbf{Polite Rejection}\\
Write a kind but firm rejection email.

\textbf{Positive Feedback}\\
Give encouraging feedback on this work.

\textbf{Client Update}\\
Draft a weekly project update email.

\textbf{Status Request}\\
Politely ask for a status update on this.

\textbf{Upsell Pitch}\\
Pitch an upgraded plan to an existing customer.

\textbf{Warm Follow-up}\\
Follow up with someone who hasn't responded.

\textbf{Conflict Resolver}\\
Write a diplomatic message to address a complaint.

\textbf{Team Praise}\\
Congratulate a team member on a job well done.

\textbf{Thank You Note}\\
Write a short, sincere thank-you email.

\textbf{Cold Outreach}\\
Introduce yourself to a potential collaborator.

gellitos

\subsection{4. Transformation}\label{transformation}

Change format, perspective, or medium while keeping meaning intact.

\textbf{Listify This}\\
Convert this paragraph into a bulleted list.

\textbf{Turn into Story}\\
Rewrite this summary as a narrative.

\textbf{Poemify}\\
Turn this text into a short poem.

\textbf{Into a Tweet}\\
Convert this into a 280-character tweet.

\textbf{As Dialogue}\\
Turn this into a conversation between two characters.

\textbf{Make It a Metaphor}\\
Describe this concept using a metaphor.

\textbf{Flip the Perspective}\\
Rewrite this from the opposite point of view.

\textbf{As a Riddle}\\
Turn this explanation into a playful riddle.

\textbf{Slide Format}\\
Reformat this as bullet points for presentation slides.

\textbf{Table-ize}\\
Convert this into a two-column comparison table.

gellitos

\subsection{5. Explanation}\label{explanation}

Clarify, demystify, and break things down step-by-step.

\textbf{Explain Like I'm 5}\\
Simplify this topic for a young child.

\textbf{Why It Matters}\\
Explain why this concept is important in plain terms.

\textbf{How It Works}\\
Break down how this mechanism operates step by step.

\textbf{What's the Catch?}\\
Explain potential drawbacks of this idea.

\textbf{Compare to a Car}\\
Use a car analogy to explain this.

\textbf{Draw Me a Timeline}\\
Present this info as a chronological sequence.

\textbf{Show Me the Math}\\
Walk through the logic or equation behind this.

\textbf{Use an Analogy}\\
Explain this concept using a creative comparison.

\textbf{From the Top}\\
Start from scratch and explain as if I know nothing.

\textbf{Summarize and Expand}\\
First summarize this in one sentence, then explain in full.

\subsubsection{6. Style Expansion}\label{style-expansion}

Shift voice, tone, and energy by cloning style.

\textbf{Make It Poetic} Rewrite this like a poem using the same idea.

\textbf{Bolder Version} Rewrite this to be more provocative.

\textbf{More Playful} Keep the structure but add playfulness.

\textbf{More Corporate} Turn this into business-speak with buzzwords.

\textbf{From Academic to Accessible} Rewrite this so a general audience
can understand.

\textbf{Spice the Voice} Make the tone more colorful and expressive.

\textbf{Flatten the Jargon} Rewrite this so it's understandable by a
layperson.

\textbf{Snark Inject} Add subtle sarcasm while keeping it intelligent.

\textbf{Charming Rewrite} Rewrite this to be warm and endearing.

\textbf{Subtle Sales Pitch} Add a persuasive undertone without being
obvious.

gellitos

\subsubsection{7. Visual Examples}\label{visual-examples}

Base the prompt on an image, layout, or formatting clue.

\textbf{Describe This Image} Write a caption for this visual.

\textbf{Emulate This Chart} Here's a table---make another that matches
the format.

\textbf{Design Prompt} Based on this layout, describe a matching design.

\textbf{Diagram It} Convert this idea into a diagram with components.

\textbf{Create Matching Icons} Based on this list, describe an icon for
each.

\textbf{Interface Copy} Write interface labels like the example shown.

\textbf{Flowchart Language} Turn this explanation into a flowchart
format.

\textbf{Side-by-Side Copy} Match the layout of this content in a new
topic.

\textbf{Describe a New Scene} Match the tone and vibe of this photo
description.

\textbf{Match a Slide} Write bullet points for a slide like the one
shown.

gellitos

\subsubsection{8. Role-Based Examples}\label{role-based-examples}

Use personas or archetypes as demonstration tools.

\textbf{Therapist Version} Respond like a therapist using this example
as a model.

\textbf{CEO Rewrite} Say this as a tech CEO would.

\textbf{Student Voice} Say this from the point of view of a college
student.

\textbf{Chef's Take} Describe this process like a professional chef.

\textbf{Professor Rewrite} Deliver this answer like a philosophy
professor.

\textbf{Startup Founder Spin} Pitch this like a founder to investors.

\textbf{Customer Service Style} Turn this into a polite helpdesk
response.

\textbf{Comedian Rewrite} Tell this like a standup comic would.

\textbf{Coach Version} Motivate like a personal trainer.

\textbf{News Anchor Version} Deliver this line like a breaking news
segment.

gellitos

\subsubsection{9. Interactive Examples}\label{interactive-examples}

Use dialogue, quiz, or engagement structure in prompt.

\textbf{Call and Response} Based on this call, write the next response.

\textbf{Quiz Me} Build a quiz that mimics this format.

\textbf{Dialogue Prompting} Continue this back-and-forth exchange.

\textbf{Correction Mode} Given this answer, write the tutor's
corrections.

\textbf{Feedback Loop} Write an example, then write feedback on it.

\textbf{Turn and Respond} Respond to this statement as if you're in the
room.

\textbf{Q\&A Pattern} Follow this format of question/answer.

\textbf{Game Style Reply} Match the tone of this text-based adventure.

\textbf{Mimic This Format} Repeat this interview Q\&A format with new
topics.

\textbf{Ping-Pong Format} Go back and forth, alternating tone with each
line.

gellitos

\subsubsection{10. Meta Examples}\label{meta-examples}

Use examples that model prompting itself.

\textbf{Prompt Anatomy} Break down this prompt into its components.

\textbf{Prompt and Output Pair} Give me both the prompt and a great
response.

\textbf{Bad Prompt Fix} Here's a weak prompt---improve it and show why.

\textbf{Prompt Variation} Write five prompts that produce similar
outputs.

\textbf{Prompt Expansion} Take this and add examples, role, and format.

\textbf{Turn Prompt to Template} Rewrite this as a reusable prompt
template.

\textbf{Add Constraints to Prompt} Modify this to include a word limit
and tone.

\textbf{Prompt Tuner} Change this prompt to get more creative answers.

\textbf{Prompt Formatter} Rewrite this in a structure that's easy to
copy/paste.

\textbf{Meta Mirror} Analyze this example and rewrite it to teach
prompting.

gellitos

\subsection{Examples}\label{examples}

Here are 100 examples prompts in 10 themed subcategories.

\subsection{1. Pattern Matching}\label{pattern-matching}

Teach the model to mimic your structure and flow.

\textbf{Follow the Form}\\
Here's one bio---generate another with the same structure:

\textbf{Mimic Style}\\
Copy this sentence's tone and cadence without repeating words:

\textbf{Repeat the Format}\\
Generate three examples in this format:

\textbf{Fill in the Blank}\\
Continue this structure logically:

\textbf{Match this Outline}\\
Use this outline and fill it in with new content:

\textbf{Repeat with Variation}\\
Produce five variations based on this input:

\textbf{Finish the Set}\\
Here are four list items---add three more that fit:

\textbf{Imitate the Opening}\\
Write a paragraph starting just like this one:

\textbf{Reflect this Rhythm}\\
Write a response with matching sentence length and flow:

\textbf{Structure Swap}\\
Use the same format but change the topic completely:

gellitos

\subsection{2. Contrast \& Correction}\label{contrast-correction}

Show bad examples, flawed drafts, and then improve them.

\textbf{Improve the Draft}\\
Here's a weak version. Rewrite it to be stronger:

\textbf{Spot the Flaws}\\
Critique this paragraph and suggest changes:

\textbf{Polish the Pitch}\\
Here's a rough idea---make it sound compelling:

\textbf{Fix the Mistakes}\\
Identify and correct grammar issues in this text:

\textbf{Upgrade the Argument}\\
Strengthen this reasoning with better logic:

\textbf{Refine the Reply}\\
This email sounds passive. Make it more confident:

\textbf{Before and After}\\
Give me the ``bad'' version and the improved one:

\textbf{Rewrite to Respect}\\
This sounds rude. Rewrite it with kindness:

\textbf{Fix the Flow}\\
This feels disjointed---make it smooth:

\textbf{Reverse the Tone}\\
Rewrite this optimistic note to be somber:

gellitos

\subsection{3. Comparative Learning}\label{comparative-learning}

Two styles, two outcomes---highlight the difference.

\textbf{Option A vs B}\\
Compare these two summaries and explain what changed:

\textbf{Vary the Ending}\\
Keep the beginning---change how it ends:

\textbf{Dual Rewrite}\\
Rewrite this both formally and casually:

\textbf{Split the Voice}\\
One version should sound human, one robotic:

\textbf{Subjective/Objective}\\
Write two takes---one personal, one analytical:

\textbf{Rewrite in Two Genres}\\
One version sci-fi, the other romance:

\textbf{Tone Flip}\\
First version skeptical, second enthusiastic:

\textbf{Change the Lens}\\
Describe this scene once as a child, once as a detective:

\textbf{Reframe the Hook}\\
One version as a mystery, one as comedy:

\textbf{Two Audiences}\\
One for beginners, one for experts:

gellitos

\subsection{4. Prompted by Example}\label{prompted-by-example}

Prompt the model with a prototype to follow.

\textbf{Write More Like This}\\
Based on this, write three more in the same tone:

\textbf{Extend the Pattern}\\
Use this example and continue the list:

\textbf{Rephrase to Match}\\
Rewrite this sentence to match the given one's rhythm:

\textbf{Style Match}\\
Write in the style of this paragraph:

\textbf{Sentence Twin}\\
Create a new sentence that mirrors this one's structure:

\textbf{Copy the Syntax}\\
Build a new example with the same grammar flow:

\textbf{Mirror the Function}\\
The purpose of this message is to inspire. Write another:

\textbf{Match the Mood}\\
This sample is tense. Write another with the same mood:

\textbf{Parallel Rewrite}\\
This sounds good. Write a new one with a new topic:

\textbf{Follow the Example}\\
Here's a Q\&A pair---give me three more like it:

gellitos

\subsection{5. Step-by-Step by Example}\label{step-by-step-by-example}

Use demonstrations to scaffold process and thought.

\textbf{Demonstrate a Fix}\\
Show how to correct this email with comments inline:

\textbf{Walkthrough Format}\\
Solve this problem like a tutorial with labeled steps:

\textbf{Inline Instruction}\\
Annotate this message with improvement tips:

\textbf{Compare Outputs}\\
Show what a weak vs strong answer looks like:

\textbf{Demonstrate Format}\\
Give an example of how to format this document:

\textbf{Template \& Fill}\\
Here's a filled template---create a blank version for reuse:

\textbf{Two-Step Rewrite}\\
First summarize, then expand it again differently:

\textbf{Explain While Doing}\\
Teach this task while performing it:

\textbf{Process vs Outcome}\\
First show the result, then how to get there:

\textbf{Live Edit}\\
Take this paragraph and revise it line-by-line:

gellitos

\subsection{6. Style Expansion}\label{style-expansion-1}

Shift voice, tone, and energy by cloning style.

\textbf{Make It Poetic}\\
Rewrite this like a poem using the same idea:

\textbf{Bolder Version}\\
Rewrite this to be more provocative:

\textbf{More Playful}\\
Keep the structure but add playfulness:

\textbf{More Corporate}\\
Turn this into business-speak with buzzwords.

\textbf{From Academic to Accessible}\\
Rewrite this so a general audience can understand.

\textbf{Spice the Voice}\\
Make the tone more colorful and expressive.

\textbf{Flatten the Jargon}\\
Rewrite this so it's understandable by a layperson.

\textbf{Snark Inject}\\
Add subtle sarcasm while keeping it intelligent.

\textbf{Charming Rewrite}\\
Rewrite this to be warm and endearing.

\textbf{Subtle Sales Pitch}\\
Add a persuasive undertone without being obvious.

gellitos

\subsection{7. Visual Examples}\label{visual-examples-1}

Base the prompt on an image, layout, or formatting clue.

\textbf{Describe This Image}\\
Write a caption for this visual.

\textbf{Emulate This Chart}\\
Here's a table---make another that matches the format.

\textbf{Design Prompt}\\
Based on this layout, describe a matching design.

\textbf{Diagram It}\\
Convert this idea into a diagram with components.

\textbf{Create Matching Icons}\\
Based on this list, describe an icon for each.

\textbf{Interface Copy}\\
Write interface labels like the example shown.

\textbf{Flowchart Language}\\
Turn this explanation into a flowchart format.

\textbf{Side-by-Side Copy}\\
Match the layout of this content in a new topic.

\textbf{Describe a New Scene}\\
Match the tone and vibe of this photo description.

\textbf{Match a Slide}\\
Write bullet points for a slide like the one shown.

gellitos

\subsection{8. Role-Based Examples}\label{role-based-examples-1}

Use personas or archetypes as demonstration tools.

\textbf{Therapist Version}\\
Respond like a therapist using this example as a model.

\textbf{CEO Rewrite}\\
Say this as a tech CEO would.

\textbf{Student Voice}\\
Say this from the point of view of a college student.

\textbf{Chef's Take}\\
Describe this process like a professional chef.

\textbf{Professor Rewrite}\\
Deliver this answer like a philosophy professor.

\textbf{Startup Founder Spin}\\
Pitch this like a founder to investors.

\textbf{Customer Service Style}\\
Turn this into a polite helpdesk response.

\textbf{Comedian Rewrite}\\
Tell this like a standup comic would.

\textbf{Coach Version}\\
Motivate like a personal trainer.

\textbf{News Anchor Version}\\
Deliver this line like a breaking news segment.

gellitos

\subsection{9. Interactive Examples}\label{interactive-examples-1}

Use dialogue, quiz, or engagement structure in prompt.

\textbf{Call and Response}\\
Based on this call, write the next response.

\textbf{Quiz Me}\\
Build a quiz that mimics this format.

\textbf{Dialogue Prompting}\\
Continue this back-and-forth exchange.

\textbf{Correction Mode}\\
Given this answer, write the tutor's corrections.

\textbf{Feedback Loop}\\
Write an example, then write feedback on it.

\textbf{Turn and Respond}\\
Respond to this statement as if you're in the room.

\textbf{Q\&A Pattern}\\
Follow this format of question/answer.

\textbf{Game Style Reply}\\
Match the tone of this text-based adventure.

\textbf{Mimic This Format}\\
Repeat this interview Q\&A format with new topics.

\textbf{Ping-Pong Format}\\
Go back and forth, alternating tone with each line.

gellitos

\subsection{10. Meta Examples}\label{meta-examples-1}

Use examples that model prompting itself.

\textbf{Prompt Anatomy}\\
Break down this prompt into its components.

\textbf{Prompt and Output Pair}\\
Give me both the prompt and a great response.

\textbf{Bad Prompt Fix}\\
Here's a weak prompt---improve it and show why.

\textbf{Prompt Variation}\\
Write five prompts that produce similar outputs.

\textbf{Prompt Expansion}\\
Take this and add examples, role, and format.

\textbf{Turn Prompt to Template}\\
Rewrite this as a reusable prompt template.

\textbf{Add Constraints to Prompt}\\
Modify this to include a word limit and tone.

\textbf{Prompt Tuner}\\
Change this prompt to get more creative answers.

\textbf{Prompt Formatter}\\
Rewrite this in a structure that's easy to copy/paste.

\textbf{Meta Mirror}\\
Analyze this example and rewrite it to teach prompting. gellitos

\subsection{Format}\label{format}

Below are 100 prompts grouped into 10 formatting styles. Use them to
bend structure to your will.

\subsection{1. Lists \& Bullets}\label{lists-bullets}

Use bullets to shape clarity, rhythm, and hierarchy.

\textbf{Top Ten List}\\
Give me a top 10 list with short descriptions.

\textbf{Bullet Point Summary}\\
Summarize this into bullets grouped by theme.

\textbf{Pros and Cons}\\
Break this down into a two-column bullet list.

\textbf{Checklist Format}\\
Create a to-do checklist from this outline.

\textbf{Quick Wins}\\
List 5 easy actions to take right now.

\textbf{Steps in Order}\\
Numbered list for how to complete this task.

\textbf{Ranking List}\\
Rank these items from most to least important.

\textbf{Alternatives List}\\
List 5 alternatives to this idea.

\textbf{Grouped Categories}\\
Organize this info into categories and sub-bullets.

\textbf{Bolded Headings}\\
Use bolded titles for each bullet group.

gellitos

\subsection{2. Tables \& Matrices}\label{tables-matrices}

Compare and clarify with rows and columns.

\textbf{Comparison Table}\\
Create a 3-column table comparing features.

\textbf{Pros, Cons, Use Cases}\\
Make a table with columns for pros, cons, and when to use.

\textbf{SWOT Table}\\
Lay this out in a SWOT matrix.

\textbf{Risk Matrix}\\
Chart likelihood vs.~impact for these risks.

\textbf{Pricing Table}\\
Format this as a pricing plan comparison.

\textbf{Timeline Table}\\
Create a timeline table with date and milestone.

\textbf{Team Roles Table}\\
Organize team roles and responsibilities in table format.

\textbf{Multi-Choice Grid}\\
Table of choices with scores in each column.

\textbf{Editable Table Format}\\
Create this in a copy/paste markdown table.

\textbf{Table with Notes}\\
Add a fourth column for notes or tips.

gellitos

\subsection{3. Outlines \& Structures}\label{outlines-structures}

Direct flow, sequence, and hierarchy with outlines.

\textbf{Basic Outline}\\
Organize this idea into a three-level outline.

\textbf{Presentation Structure}\\
Format this as a 5-slide presentation.

\textbf{Narrative Arc}\\
Turn this into a beginning, middle, and end.

\textbf{Essay Framework}\\
Structure this into intro, argument, counterpoint, and conclusion.

\textbf{Process Steps}\\
Outline the full workflow for this task.

\textbf{Hierarchy Tree}\\
Display this as a nested structure.

\textbf{Course Outline}\\
Build a syllabus-style layout for this topic.

\textbf{Workshop Flow}\\
Plan a workshop session with time blocks.

\textbf{Script Breakdown}\\
Break this into scenes or chapters.

\textbf{Milestone Plan}\\
List key deliverables with dates and owners.

gellitos

\subsection{4. Slide \& Deck Formats}\label{slide-deck-formats}

Convert ideas into presentations and slide-friendly formats.

\textbf{Slide Bullets}\\
Convert this into bullets for a single slide.

\textbf{Slide Titles}\\
Create 5 strong titles for a presentation deck.

\textbf{Slide Order}\\
Rearrange this info into a compelling slide order.

\textbf{One-Slide Summary}\\
Reduce this to one powerful summary slide.

\textbf{Problem--Solution Slide}\\
Build a two-part slide: challenge and fix.

\textbf{Pitch Deck Format}\\
Outline a 10-slide startup pitch.

\textbf{Headline + Bullet}\\
For each idea, generate a slide title and 3 bullets.

\textbf{Big Idea Slide}\\
Frame this idea as a single bold slide.

\textbf{Side-by-Side Comparison}\\
Format this for a two-column slide.

\textbf{Final Slide CTA}\\
Write a compelling closing slide with a call to action.

gellitos

\subsection{5. Code \& Logic Structures}\label{code-logic-structures}

Shape outputs with precision for devs and tech thinkers.

\textbf{JSON Format}\\
Structure this content as a JSON object.

\textbf{Python Dict}\\
Format these variables in Python dictionary style.

\textbf{If--Then Logic}\\
Turn this into a list of if/then statements.

\textbf{Decision Tree Format}\\
Structure this logic as a branching decision tree.

\textbf{API Request Format}\\
Create a sample API request with headers and parameters.

\textbf{Markdown Format}\\
Convert this into clean markdown syntax.

\textbf{Data Table}\\
Output this info as a CSV-style table.

\textbf{Regex Examples}\\
Generate regex patterns based on this text.

\textbf{YAML Format}\\
Write this structured prompt as YAML.

\textbf{Code Comment Format}\\
Annotate this code block with inline comments.

gellitos

\subsection{6. Question \& Answer
Formats}\label{question-answer-formats}

Format as interviews, FAQs, and conversational flows.

\textbf{FAQ Style}\\
Turn this into a frequently asked questions list.

\textbf{Interview Format}\\
Write this as a Q\&A interview.

\textbf{Quiz Format}\\
Structure this content as a multiple-choice quiz.

\textbf{Socratic Dialogue}\\
Create a question-driven conversation.

\textbf{Rapid Fire Q\&A}\\
Make this a quick back-and-forth sequence.

\textbf{Explainer with Questions}\\
Interweave questions and answers.

\textbf{Three-Tiered Answer}\\
Provide a short, medium, and long answer.

\textbf{Inquisitive Prompt}\\
Write a question designed to provoke deep insight.

\textbf{Structured Inquiry}\\
Use numbered questions to guide learning.

\textbf{Conversational Coach}\\
Format this as a guided dialogue with responses.

gellitos

\subsection{7. Email \& Message Styles}\label{email-message-styles}

Shape output as professional or casual messages.

\textbf{Formal Email}\\
Write this as a polished business email.

\textbf{Casual Message}\\
Turn this into a friendly Slack message.

\textbf{Follow-up Email}\\
Write a polite follow-up on this topic.

\textbf{Thank You Email}\\
Format a warm and professional thank-you.

\textbf{Customer Support Email}\\
Turn this into a support response email.

\textbf{Cold Outreach}\\
Write an email introducing yourself to a stranger.

\textbf{Apology Email}\\
Craft a sincere and effective apology message.

\textbf{Sales Email}\\
Format this into a persuasive cold pitch.

\textbf{Internal Update}\\
Convert this into a team status update.

\textbf{Auto-Reply Message}\\
Write an out-of-office or away message.

gellitos

\subsection{8. Web \& Content Layouts}\label{web-content-layouts}

Turn ideas into web-friendly blocks.

\textbf{Landing Page Format}\\
Build a landing page layout from this copy.

\textbf{Hero + CTA}\\
Write a hero headline and subhead with CTA.

\textbf{Blog Post Sections}\\
Break this into H2/H3 structure for a post.

\textbf{Sidebar Format}\\
Reformat key ideas as sidebar callouts.

\textbf{Card Layout}\\
Describe this idea in 3 stacked cards.

\textbf{Feature Section}\\
Highlight this product's features with icons and bullets.

\textbf{Pricing Page Copy}\\
Write copy for 3 pricing tiers with benefits.

\textbf{Callout Boxes}\\
Turn these tips into callout blocks.

\textbf{Grid Format}\\
Layout this idea as a 3x3 grid of content.

\textbf{Page Wireframe}\\
Describe a full-page content structure.

gellitos

\subsection{9. Scripts \& Performance
Formats}\label{scripts-performance-formats}

Build scenes, speeches, and roleplay instructions.

\textbf{Script Format}\\
Write this as a character script.

\textbf{Two-Person Dialogue}\\
Turn this into a back-and-forth scene.

\textbf{Speech Format}\\
Format this as a short speech.

\textbf{Monologue Style}\\
Express this as an internal monologue.

\textbf{Narration Block}\\
Write this as voiceover narration.

\textbf{Stage Directions}\\
Add scene directions and cues.

\textbf{Interactive Script}\\
Format this like a choose-your-own-path interaction.

\textbf{Performance Pitch}\\
Write a pitch with stage presence.

\textbf{Podcast Script}\\
Convert this idea into a podcast outline.

\textbf{Host + Guest Format}\\
Create a segment for an interview show.

gellitos

\subsection{10. Mixed Media Formats}\label{mixed-media-formats}

Go cross-modal: text plus visuals, audio, and interaction.

\textbf{Image Prompt Format}\\
Write a prompt for generating this image.

\textbf{Video Description}\\
Describe the shot list for this video idea.

\textbf{Audio Script}\\
Create an audio ad script with timestamps.

\textbf{Interactive Layout}\\
Design an interactive page structure.

\textbf{Visual Storyboard}\\
Build a 4-panel storyboard description.

\textbf{AR/VR Prompt}\\
Write an immersive prompt for spatial experience.

\textbf{Story + Sound}\\
Combine a scene with suggested soundtrack.

\textbf{Printable Format}\\
Turn this into a printable worksheet.

\textbf{Poster Copy}\\
Write copy for a striking poster layout.

\textbf{Mixed Format Prompt}\\
Combine text, image, and interaction in one prompt.

gellitos

\subsection{Context}\label{context}

Below are 100 context prompts grouped into 10 subcategories.

\subsection{1. Purpose Framing}\label{purpose-framing}

Clarify the goal behind the prompt.

\textbf{Summarize to Inform}\\
Write a summary that helps someone make a decision.

\textbf{Convert to Persuade}\\
Make this compelling for a skeptical audience.

\textbf{Explain to Teach}\\
Turn this into a short lesson for a beginner.

\textbf{Inspire Action}\\
Frame this message to motivate behavior.

\textbf{Report with Authority}\\
Summarize these findings with professional confidence.

\textbf{Persuade Gently}\\
Soften the message while keeping it persuasive.

\textbf{Encourage Exploration}\\
Make this sound like an invitation to explore.

\textbf{Frame as Advice}\\
Position this as helpful peer guidance.

\textbf{Reduce Resistance}\\
Present this idea in a way that's easy to accept.

\textbf{Pitch Softly}\\
Make this persuasive without sounding like a hard sell.

gellitos

\subsection{2. Audience Targeting}\label{audience-targeting}

Shape the tone and language for the right listener.

\textbf{Write for Teens}\\
Rephrase this for a teenage audience.

\textbf{Make it Executive-Friendly}\\
Adapt this for a C-level reader.

\textbf{Explain to My Grandma}\\
Make this accessible to someone unfamiliar with the tech.

\textbf{Technical Peer Mode}\\
Write this for a fellow expert.

\textbf{Investor Audience}\\
Shape this as a pitch to investors.

\textbf{Team Briefing Format}\\
Write this like a project manager updating the team.

\textbf{Customer-Facing Tone}\\
Reword for external business communication.

\textbf{Young Reader Voice}\\
Reframe for middle school comprehension.

\textbf{Informed But Busy}\\
Communicate key points quickly and clearly.

\textbf{Reader is Suspicious}\\
Write this as if the reader doesn't trust the source.

gellitos

\subsection{3. Emotional Tone}\label{emotional-tone}

Set the emotional backdrop for the output.

\textbf{Make It Uplifting}\\
Rewrite this to leave the reader feeling hopeful.

\textbf{Add Empathy}\\
Include understanding for the reader's situation.

\textbf{Neutralize Drama}\\
Tone down emotional intensity while staying clear.

\textbf{Reassure the Reader}\\
Make this sound comforting and safe.

\textbf{Inspire Confidence}\\
Use language that builds trust and certainty.

\textbf{Amplify Urgency}\\
Add emotional tension to prompt fast action.

\textbf{Show Humility}\\
Make this message more modest and relatable.

\textbf{Sound More Human}\\
Rewrite this with emotional nuance.

\textbf{Add Compassion}\\
Show concern for the person's experience.

\textbf{Positive Framing}\\
Recast the message in a more optimistic light.

gellitos

\subsection{4. Relevance \& Background}\label{relevance-background}

Give the model a situation to respond from.

\textbf{Add Context Brief}\\
Use this background before generating output.

\textbf{Frame with History}\\
Add relevant historical information.

\textbf{Explain the Stakes}\\
Clarify what's at risk or why it matters.

\textbf{Who's Involved}\\
Describe the key players before the model replies.

\textbf{Include Timeline}\\
Insert a short timeline for context.

\textbf{Set the Scene}\\
Describe the situation as setup for the prompt.

\textbf{Mention Location}\\
Add where this is happening to guide tone.

\textbf{What's Been Tried}\\
Briefly describe past efforts or attempts.

\textbf{Clarify Intentions}\\
State why the user is asking.

\textbf{Include Known Constraints}\\
Set limitations as part of the initial framing.

gellitos

\subsection{5. Framing as Scenarios}\label{framing-as-scenarios}

Contextualize prompts as situations or use cases.

\textbf{Case Study Setup}\\
Frame this as an example from a real project.

\textbf{Tell It as a Story}\\
Turn this into a use case narrative.

\textbf{Fictional Scenario}\\
Invent a situation where this problem arises.

\textbf{Day in the Life}\\
Write this from someone's daily experience.

\textbf{If--Then Frame}\\
If this happens, what should follow?

\textbf{Critical Moment}\\
Describe a high-stakes point in time.

\textbf{What Would Happen If}\\
Explore a hypothetical situation.

\textbf{Emergency Response}\\
Frame this as a crisis requiring urgent response.

\textbf{Team Conflict Setup}\\
Describe a team disagreement as setup.

\textbf{Pitch Scenario}\\
Set the context as a make-or-break presentation.

gellitos

\subsection{6. Time \& Place Cues}\label{time-place-cues}

Situate output in a specific moment.

\textbf{Frame as Breaking News}\\
Write this with urgency like it just happened.

\textbf{End-of-Year Wrap}\\
Reflect on the year and preview what's next.

\textbf{First Day Framing}\\
Frame this as someone's first time experiencing it.

\textbf{Yesterday--Today--Tomorrow}\\
Use temporal structure to build the message.

\textbf{90-Day Plan}\\
Lay out actions over a 3-month horizon.

\textbf{Real-Time Style}\\
Make it sound like it's happening right now.

\textbf{Five Years Later}\\
Frame as a future reflection.

\textbf{Countdown Format}\\
Use a countdown to frame urgency.

\textbf{Annual Report Voice}\\
Use formal year-in-review language.

\textbf{Mid-Project Update}\\
Situate this as an in-progress check-in.

gellitos

\subsection{7. Communication Channel
Framing}\label{communication-channel-framing}

Tailor output to the platform or format.

\textbf{LinkedIn Post Style}\\
Reformat this as a professional LinkedIn update.

\textbf{Slack Message Tone}\\
Make this brief and informal like a team chat.

\textbf{Newsletter Edition}\\
Frame as part of a recurring newsletter.

\textbf{Internal Memo}\\
Write this as a confidential staff memo.

\textbf{Presentation Slide Notes}\\
Make this text slide-speaker ready.

\textbf{Meeting Agenda Format}\\
Turn this into a clear meeting outline.

\textbf{Voiceover Style}\\
Write this as if for an audio narration.

\textbf{Podcast Teaser}\\
Summarize this to hook podcast listeners.

\textbf{YouTube Description}\\
Format this as a video description with links.

\textbf{Job Posting Format}\\
Frame this content as a job description.

gellitos

\subsection{8. User Intent Cues}\label{user-intent-cues}

Focus the model by clarifying why the user is prompting.

\textbf{I Want to Learn}\\
Frame this as a learning prompt for the user.

\textbf{I Need to Decide}\\
Present output that helps a user choose.

\textbf{I'm Exploring}\\
Make this open-ended and curiosity-driven.

\textbf{I Need Help}\\
Respond with tone appropriate to a help request.

\textbf{I'm Researching}\\
Frame the reply like a summary of findings.

\textbf{I'm Making a Case}\\
Assist in building a logical argument.

\textbf{I'm Writing Something}\\
Treat this as a co-writing session.

\textbf{I'm Explaining to Others}\\
Format this to make teaching easier.

\textbf{I'm Pitching}\\
Focus this prompt on persuasion and clarity.

\textbf{I'm Feeling Overwhelmed}\\
Write this with empathy and clarity.

gellitos

\subsection{9. Psychological Framing}\label{psychological-framing}

Tweak assumptions, attitudes, or energy.

\textbf{Scarcity Mentality}\\
Write with a sense of limited opportunity.

\textbf{Abundance Mentality}\\
Emphasize plenty, options, and creativity.

\textbf{Fixed vs Growth Mindset}\\
Reframe this with a learning-forward lens.

\textbf{Empowerment Language}\\
Make this sound energizing and empowering.

\textbf{Reverse Psychology}\\
Try saying the opposite to provoke thought.

\textbf{Challenge the Reader}\\
Use tone that pushes for improvement.

\textbf{Unblock the Blocked}\\
Encourage action from a stuck mindset.

\textbf{Default to Optimism}\\
Reframe setbacks as potential.

\textbf{Invite Curiosity}\\
Make this message spark exploration.

\textbf{Tone Down Ego}\\
Rewrite this with more humility.

gellitos

\subsection{10. Framing with Memory or
Identity}\label{framing-with-memory-or-identity}

Use backstory or persona to steer behavior.

\textbf{Refer to Past Action}\\
Mention what the user already did.

\textbf{Mention Past Success}\\
Build on earlier accomplishments.

\textbf{Frame as Identity}\\
Respond as if the user is a designer/writer/leader/etc.

\textbf{Use Personal History}\\
Reference earlier parts of the conversation.

\textbf{Mention Habits}\\
Talk about routines or repeated behavior.

\textbf{Frame by Values}\\
Anchor this in what matters to the person.

\textbf{Invoke Prior Mistakes}\\
Gently mention what didn't work before.

\textbf{Tap a Self-Image}\\
Respond as if the person sees themselves a certain way.

\textbf{Custom Context Setting}\\
Let the user write their own backstory as framing.

\textbf{Memory-Driven Prompt}\\
Summon something remembered to steer the tone. gellitos

\subsection{Role}\label{role}

This chapter gives you 50 role-based prompts. Each subcategory reflects
a different kind of role: from professional personas to creative
characters.

\subsection{1. Professional Roles}\label{professional-roles}

Assign the model an expert identity.

\textbf{Be a Copywriter}\\
Act like a marketing copywriter and rewrite this message.

\textbf{Be a Consultant}\\
Advise me like a strategic business consultant.

\textbf{Be a Lawyer}\\
Draft this like an attorney preparing a brief.

\textbf{Be a Therapist}\\
Respond with empathy like a licensed therapist.

\textbf{Be a Doctor}\\
Explain this like a general practitioner to a curious patient.

\textbf{Be an Economist}\\
Analyze this from a macroeconomic perspective.

\textbf{Be a Project Manager}\\
Break this into tasks with timelines and owners.

\textbf{Be a Data Scientist}\\
Interpret this information like a data analyst.

\textbf{Be a UX Designer}\\
Critique this from a user experience angle.

\textbf{Be a Teacher}\\
Explain it like a passionate educator.

gellitos

\subsection{2. Creator Roles}\label{creator-roles}

Tell the model to create like an artist, writer, or storyteller.

\textbf{Be a Poet}\\
Rewrite this like a free verse poem.

\textbf{Be a Sci-Fi Author}\\
Turn this prompt into a worldbuilding snippet.

\textbf{Be a Stand-Up Comedian}\\
Make this funny with a dry sense of humor.

\textbf{Be a Songwriter}\\
Turn this text into song lyrics.

\textbf{Be a Screenwriter}\\
Write this as if it's a scene in a film.

\textbf{Be a Myth Maker}\\
Turn this concept into a modern fable.

\textbf{Be a Horror Author}\\
Rewrite this to send chills down the spine.

\textbf{Be a Children's Book Author}\\
Make this suitable for young readers.

\textbf{Be a Graphic Novelist}\\
Write this with visual panels in mind.

\textbf{Be a Dungeon Master}\\
Turn this into a branching adventure setup.

gellitos

\subsection{3. Public Roles}\label{public-roles}

Shape the voice with persona of influence.

\textbf{Be a Politician}\\
Reframe this message for public support.

\textbf{Be a Spokesperson}\\
Deliver this as a statement to the press.

\textbf{Be a CEO}\\
Frame this as an executive announcement.

\textbf{Be a Coach}\\
Respond like a personal development coach.

\textbf{Be a Journalist}\\
Rewrite this as a breaking news report.

\textbf{Be a Historian}\\
Analyze this through a historical lens.

\textbf{Be a Futurist}\\
Project future scenarios from this trend.

\textbf{Be a Scientist}\\
Explain this through evidence and research.

\textbf{Be a Philosopher}\\
Interpret this as an ethical or abstract idea.

\textbf{Be a Debate Moderator}\\
Present both sides of this argument fairly.

gellitos

\subsection{4. Everyday Roles}\label{everyday-roles}

Use friendly, casual, or relatable personas.

\textbf{Be My Friend}\\
Respond like someone who knows me well.

\textbf{Be My Mentor}\\
Offer thoughtful guidance as a career mentor.

\textbf{Be My Barista}\\
Describe this idea like a clever coffee shop convo.

\textbf{Be My Roommate}\\
Keep this casual, funny, and short.

\textbf{Be My Grandparent}\\
Share this with warmth, care, and simple words.

\textbf{Be a Customer Support Agent}\\
Answer this as if on a support call.

\textbf{Be a Librarian}\\
Suggest resources in a helpful tone.

\textbf{Be a Neighbor}\\
Keep this friendly, familiar, and helpful.

\textbf{Be a Parent}\\
Respond with patience, clarity, and protection.

\textbf{Be a Pet}\\
Write a cute reply like a cat or dog might.

gellitos

\subsection{5. Meta Roles}\label{meta-roles}

Give the model instructions about its \emph{own} behavior.

\textbf{Be Self-Aware}\\
Think out loud as you generate this output.

\textbf{Be My Thinking Partner}\\
Ask clarifying questions before you answer.

\textbf{Be Curious}\\
Suggest new directions I hadn't considered.

\textbf{Be Critical}\\
Challenge my assumptions constructively.

\textbf{Be a Contrarian}\\
Argue the opposite with logic and grace.

\textbf{Be Playful}\\
Add wit and whimsy to this reply.

\textbf{Be Precise}\\
Focus on clarity, accuracy, and rigor.

\textbf{Be Creative}\\
Take the idea in an unexpected direction.

\textbf{Be Concise}\\
Say the most with the fewest words.

\textbf{Be Slow and Careful}\\
Walk through the logic step by step.

gellitos

\subsection{Constraints}\label{constraints}

Below are 50 prompts grouped into 5 constraint types.

\subsection{1. Word \& Length Limits}\label{word-length-limits}

Control output with precision and brevity.

\textbf{10-Word Max}\\
Summarize this in no more than 10 words.

\textbf{One Sentence Only}\\
Say this in a single complete sentence.

\textbf{50-Word Summary}\\
Limit the response to exactly 50 words.

\textbf{Tweet Format}\\
Fit this response under 280 characters.

\textbf{Three-Sentence Rule}\\
Answer in no more than 3 full sentences.

\textbf{Haiku Constraint}\\
Reply in the structure of a traditional haiku.

\textbf{Six Words}\\
Describe this in exactly six words.

\textbf{One Paragraph Only}\\
Make your point in one short paragraph.

\textbf{Limited Vocabulary}\\
Use only the 1000 most common English words.

\textbf{Brief Reply}\\
Say this in the shortest complete form.

gellitos

\subsection{2. Structural Rules}\label{structural-rules}

Impose formatting and rhythm-based patterns.

\textbf{List of 3}\\
Respond with exactly three bullet points.

\textbf{Acronym Reply}\\
Format this as an acronym and define each letter.

\textbf{Rhyming Format}\\
Make each sentence end in a rhyme.

\textbf{Alphabetical Answer}\\
Each line starts with the next letter of the alphabet.

\textbf{Double-Spaced}\\
Add a blank line between every line of text.

\textbf{Chronological Order}\\
Arrange this in order of time.

\textbf{Table Form Only}\\
Present this only in table format.

\textbf{Step-by-Step Only}\\
Write a step-by-step guide with clear numbers.

\textbf{Question Format Only}\\
Rewrite this entirely as a set of questions.

\textbf{One-Line Format}\\
Each idea must be a single line.

gellitos

\subsection{3. Style Constraints}\label{style-constraints}

Impose tone, personality, and diction rules.

\textbf{Alliteration Rule}\\
Use alliteration heavily in your response.

\textbf{No Adjectives}\\
Avoid all descriptive modifiers.

\textbf{Metaphor Only}\\
Explain using only metaphor and analogy.

\textbf{No Questions}\\
Don't ask anything---just explain.

\textbf{Passive Voice Only}\\
Rewrite the message in passive voice.

\textbf{No First Person}\\
Remove all uses of ``I'' and ``we.''

\textbf{Formal Academic Tone}\\
Use scholarly language only.

\textbf{Super Casual Tone}\\
Keep it extremely laid back and chill.

\textbf{No Buzzwords}\\
Eliminate jargon and trendy phrases.

\textbf{No Pronouns}\\
Write the entire reply without pronouns.

gellitos

\subsection{4. Time Constraints}\label{time-constraints}

Force brevity through urgency or real-time limits.

\textbf{60-Second Read}\\
Write only what can be read in under a minute.

\textbf{Flash Summary}\\
Condense into something skimmable in 10 seconds.

\textbf{Rapid Bullet Version}\\
List five points as quickly as possible.

\textbf{Short Talk Format}\\
Write as if you're giving a 2-minute talk.

\textbf{Speed Run}\\
Describe this topic in lightning-fast form.

\textbf{Response in 3 Beats}\\
Give just a beginning, middle, and end.

\textbf{One Breath Rule}\\
Make it possible to say this in one breath.

\textbf{No Time to Explain}\\
Write like you only have one shot to say it.

\textbf{Tweetstorm Lite}\\
Deliver this in three linked micro-posts.

\textbf{60 Words Max}\\
Do not exceed 60 words in total.

gellitos

\subsection{5. Creative Constraints}\label{creative-constraints}

Add odd, playful, or quirky restrictions.

\textbf{Emoji Limit}\\
You can use no more than three emojis.

\textbf{No ``E'' Rule}\\
Avoid the letter ``e'' in the entire response.

\textbf{Only Questions}\\
Reply using only interrogative statements.

\textbf{Two Characters Arguing}\\
Make this a debate between two voices.

\textbf{Reverse Logic}\\
Explain by stating what it's not.

\textbf{Mirror Sentence Rule}\\
Each sentence must be a reverse of the last.

\textbf{Limiting POV}\\
Only describe what a blind person would notice.

\textbf{Opposites Only}\\
Define something by explaining its opposite.

\textbf{Sound-Only Description}\\
Describe the thing using only sounds and onomatopoeia.

\textbf{Color-Limited Prompt}\\
Use only metaphors related to a single color.

gellitos

\section{Appendix: Prompt Outside the
Box}\label{appendix-prompt-outside-the-box}

Here lies the playground. The edges. The uncategorizable and
delightfully weird. These prompts don't fit the six classic elements,
and that's exactly why they belong here.

Use them when you want to provoke something unexpected.

\subsection{1. Wild Cards}\label{wild-cards}

Prompting the unexpected.

\textbf{Invent a New Prompt Format}\\
Make up a new kind of prompt no one's used yet.

\textbf{Prompt About Prompts}\\
Reflect on how prompt structure affects thinking.

\textbf{Turn This Into a Ritual}\\
Create a daily prompt-based ritual.

\textbf{Prompt Reverser}\\
Take this output and guess what the original prompt was.

\textbf{Time Capsule Prompt}\\
Write a prompt for someone 100 years from now.

\textbf{Prompt Loop}\\
Create a prompt that improves itself on each reply.

\textbf{Recursive Prompt}\\
Ask the assistant to write a prompt that asks for a prompt.

\textbf{Dreamstate Prompt}\\
Format this like a dream journal entry.

\textbf{Prompt as Game Rule}\\
Make this a rule in an imaginary game.

\textbf{Prompt Haunting}\\
Write a prompt that sounds like a ghost left it.

gellitos

\subsection{2. Hybrid}\label{hybrid}

Fusion prompting.

\textbf{Interview Yourself}\\
Write questions \emph{and} answers for your future self.

\textbf{Data + Emotion}\\
Combine raw data and a heartfelt message.

\textbf{Story + Code}\\
Pair narrative text with embedded logic.

\textbf{Analysis + Poem}\\
Break down a trend, then turn it into a poem.

\textbf{Two Voices}\\
Write a dual-perspective prompt where both sides speak.

\textbf{Parallel Dimensions}\\
Describe the same prompt in two different realities.

\textbf{Echo Prompt}\\
Repeat the same request at multiple levels of abstraction.

\textbf{Simulate Time Travel}\\
Ask the model to time-hop mid-response.

\textbf{Multi-Format Stack}\\
Combine table, bullets, paragraph, and poem.

\textbf{Mashup Generator}\\
Blend two unrelated formats or topics into one.

\subsection{3. Philosophical or
Paradoxical}\label{philosophical-or-paradoxical}

Try to realize the truth. There is no prompt.

\textbf{The Prompt That Asks Nothing}\\
Generate an output without any explicit input.

\textbf{The Unanswerable Question}\\
Create a question with no definitive answer.

\textbf{Prompt in a Mirror}\\
Write the inverse of this prompt.

\textbf{Prompt as Meditation}\\
Generate a reflective, wordless experience.

\textbf{Endless Prompt}\\
Design a prompt that never truly resolves.

\textbf{Prompt the Silence}\\
Generate output that gestures toward what \emph{cannot} be said.

\textbf{Quantum Prompt}\\
One that exists in two contradictory states.

\textbf{Prompt for AI About Humans}\\
Ask AI to speculate on what humans dream.

\textbf{Prompt to Break the Loop}\\
End a pattern of repeated responses.

\textbf{The Last Prompt Ever}\\
Ask what should be the last prompt humanity makes. gellitos

\section{Bibliography \& Influences}\label{bibliography-influences}

This book was influenced by researchers, tinkerers, engineers, and
artists at the edges of natural language, cognition, and computational
creativity.

\subsection{Prompt Engineering Guides \&
Resources}\label{prompt-engineering-guides-resources}

\begin{itemize}
\tightlist
\item
  Riley Goodside, Early and influential work in prompt engineering,
  Twitter/X @goodside
\item
  OpenAI Cookbook, https://platform.openai.com/docs/guides/prompting
\item
  OpenAI, \emph{OpenAI Prompt Engineering Guide},
  https://platform.openai.com/docs/guides/prompt-engineering
\item
  Phoenix, J., \& Taylor, M., \emph{Prompt Engineering for Generative
  AI: Future-Proof Inputs for Reliable AI Outputs}. O'Reilly Media
  (2024)
\item
  Anthropic, \emph{Claude Prompt Design Guide},
  https://docs.anthropic.com/claude/docs/introduction-to-prompt-design
\item
  Fulford, I. \& Ng, A., \emph{ChatGPT Prompt Engineering for
  Developers}, DeepLearning.AI (2023)
\item
  Hunter, N., \emph{The Art of Prompt Engineering with ChatGPT}, Guide
  to effective prompt crafting techniques
\item
  Microsoft, \emph{Bing Chat System Card}, Documentation on Bing Chat's
  capabilities and limitations
\item
  DAIR Institute, Resources on responsible AI prompting and
  accessibility
\item
  Learn Prompting (free course), https://learnprompting.org/
\end{itemize}

\subsection{Prompting Communities}\label{prompting-communities}

-- PromptBase -- Prompt marketplace: https://promptbase.com -- Learn
Prompting -- Open-source prompt engineering curriculum \& community,
Discord + https://learnprompting.org -- FlowGPT -- Prompt sharing and
discovery, Discord + https://flowgpt.com -- PromptStacks (Discord) --
Prompt productivity and GPT automation community:
https://discord.gg/promptstacks -- Civitai -- Generative AI image models
and prompts, Discord + https://civitai.com -- Prompt Engineering Hub,
Discord + https://discord.gg/promptengineering -- The Prompt Index
(Telegram) -- Curated prompt archive: https://t.me/ThePromptIndex --
PromptHero -- Prompt search and publishing for AI art:
https://prompthero.com -- Midjourney \& DALL·E communities -- AI art
prompt crafting spaces (various Discords and forums) -- Reddit:
r/PromptEngineering -- Prompting strategies and tools:
https://www.reddit.com/r/PromptEngineering -- Reddit:
r/ChatGPTPromptGenius -- Community-submitted ChatGPT prompts:
https://www.reddit.com/r/ChatGPTPromptGenius -- Reddit: r/PromptDesign
-- Prompt formatting and style discussion:
https://www.reddit.com/r/PromptDesign - Twitter / X -- @floeriam, Flo
Crivello -- GenAI founder -- @swyx, Shawn Wang -- LLM commentary and
tooling -- @markryan, Prompt education \& tool reviews

\subsection{AI Research}\label{ai-research}

\begin{itemize}
\tightlist
\item
  Brown, T., \emph{et al.} \emph{Language Models Are Few-Shot Learners}.
  OpenAI (2020) -- https://arxiv.org/abs/2005.14165
\item
  Ouyang, L., \emph{et al.} \emph{Training Language Models to Follow
  Instructions with Human Feedback}. OpenAI (2022) --
  https://arxiv.org/abs/2203.02155
\item
  Bai, Y., \emph{et al.} \emph{Constitutional AI: Harmlessness from AI
  Feedback}. Anthropic (2022) -- https://arxiv.org/abs/2212.08073
\item
  Wei, J., \emph{et al.} \emph{Chain of Thought Prompting Elicits
  Reasoning in Large Language Models}. Google Research (2022) --
  https://arxiv.org/abs/2201.11903
\end{itemize}

gellitos

\section{Index}\label{index}

\textbf{action verbs} -- 1, 3, 4, 35, 78\\
\textbf{constraint} -- 6, 211, 213, 407, 532\\
\textbf{context} -- 5, 12, 198, 203, 400\\
\textbf{example} -- 4, 120, 127, 135, 148\\
\textbf{format} -- 7, 18, 255, 272, 285\\
\textbf{persona} -- 6, 305, 308, 317, 329\\
\textbf{prompt design} -- 1, 2, 3, 80, 98\\
\textbf{structure} -- 22, 133, 243, 366, 420\\
\textbf{tone} -- 14, 55, 138, 212, 336\\
\textbf{voice} -- 8, 44, 139, 276, 315

gellitos

\section{Epilogue}\label{epilogue}

\emph{If you want the model to respond like it knows what you're talking
about, you have to tell it what you're talking about.}

gellitos

\section{Teaser}\label{teaser}

\subsubsection{\texorpdfstring{\emph{ChatGPT for Business: How to Create
Powerful AI
Workflows}}{ChatGPT for Business: How to Create Powerful AI Workflows}}\label{chatgpt-for-business-how-to-create-powerful-ai-workflows}

From the author of \emph{Prompt Power} comes a groundbreaking guide for
leaders, builders, and change-makers:

\emph{ChatGPT for Business} unveils how organizations across industries
are transforming everyday tasks into intelligent workflows using
Generative AI. Bestselling author Dan Hermes shows how companies are
leveraging Large Language Models(LLM) to transform operations,
supercharge productivity, and build unforgettable customer experiences.

Through vivid case studies grounded in real-world applications, you'll
learn how GPTs are being used across industries---from onboarding and
legal review to product development and executive strategy. Each story
is paired with clear, actionable prompts you can use today to refresh
your thinking.

You'll discover how AI can:

\begin{itemize}
\tightlist
\item
  Enhance customer empathy at scale
\item
  Streamline operations without a single line of code
\item
  Turn messy data into actionable insight
\item
  Fuel innovation with human-AI design loops
\item
  Reduce risk while increasing transparency
\end{itemize}

With a unique blend of scenarios, strategies, and prompts, this book
helps you bridge the gap between curious inquiry and practical
understanding. Whether you're a startup founder, team lead, or
enterprise executive, \emph{ChatGPT for Business} offers a hands-on path
to AI fluency.

\section{About Lexicon Systems}\label{about-lexicon-systems}

Lexicon Systems is a boutique AI consultancy and agent development firm
founded by Dan Hermes, author of Prompt Power and ChatGPT for Business.
We help companies harness the power of Generative AI, LLMs, and
intelligent automation to build successful agents and enact their AI
transformation.

For over two decades, Lexicon has delivered award-winning tech to
clients ranging from global enterprises to fast-rising startups. Our
clients include Microsoft, Accenture, Fidelity Investments, Draftkings,
Pluralsight, and the Federal Aviation Administration (FAA).

Whether you're scaling AI capability, building agents, or upskilling
your teams, Lexicon Systems has the background and knowledge to move you
forward.

Practical AI. Measurable impact. Experience.

Learn more at lexicon.systems. Yes, that's lexicon dot systems.

\end{document}
